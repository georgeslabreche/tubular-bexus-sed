\subsection{Experiment Setup} \label{Experiment_Setup}

The experiment consists of AAC sixteen sampling bags subsystem, and the CAC coiled tube subsystem. The principal aim is to validate the AAC sampling method and to do so, it is necessary to sample during descent phase in order to compare the results with the ones obtained from the CAC.

The primary concern regarding the AAC air sampling subsystem is that after cut-off the gondola will tumble and fall at an average speed of 50 m/s for approximately 2 minutes \cite{BexusManual}. This falling speed is too fast in order to sample air at the desired vertical resolution that is targeted to be 500m. This means that only after the gondola is stabilized at a descent rate of 8 m/s the sampling can be done. The tumbling phase will span approximately for 6km and considering a floating phase at 25km, the sampling can be started from 19km in altitude. Nevertheless, the main region of interest is the stratosphere, especially between 19km and 25km of altitude. It is for this reason that the team has decided to sample during ascent phase as well. Sixteen sampling bags will be filled up, ten during ascent phase approximately between 18-24km, and six during descent phase below 19km. The desired vertical resolution is 500 m at a falling speed of 8 m/s which means that using bags with a volume of 3L, an air pump that assures 3L/min intake rate is necessary for the sampling bags.

The AAC will need an air pump for sampling, due to low ambient pressure at stratospheric altitudes. The air pump is also needed in order to assure the intake flow rate and obtain a good resolution. A control valve will be used to flush the pump after each bag is filled and make sure that the next bag will be filled with fresh air from the corresponding altitude. Each sampling bag will be assigned a 500 meter altitude sampling range from which to collect air samples. At an ascent speed of 5 m/s during the Ascent Phase and at a descent speed of 8 m/s during the Descent Phase, the 3L sampling bags need to be filled by an air pump that assures 3L/min intake rate.

Shortly after the launch, all the valves in both subsystems will open in order to flush the inert gases inside the AAC and the CAC. The CAC valve will remain open at all time during ascent, floating, and descent phases. The tube will empty itself due to pressure gradient during the ascent phase and it will be filled passively during descent. The valve will close just after hitting the ground in order to preserve the sample. The AAC will need an air pump for sampling, due to low ambient pressure. The air pump is also needed in order to assure the intake flow rate and obtain a good resolution.

After sampling for a given bag is complete, the pump will be flushed and prior to the subsequent sampling bag valve being opened. This process continues until the last sampling bag is filled.This procedure occurs twice, the first time during the Ascent Phase for the 10 first sampling bags and the second time during the descent phase for the remaining 6 sampling bags.

The emptying and sampling sequence is represented in Figures \ref{fig:ascent} and \ref{fig:descent}.

\begin{figure}[H]
    \begin{align*}
        \includegraphics[width=1\linewidth]{4-experiment-design/img/ascent-phase.png}
    \end{align*}
    \caption{The emptying and sampling sequence-Ascent Phase\label{fig:ascent}}
\end{figure}

\begin{figure}[H]
    \begin{align*}
        \includegraphics[width=1\linewidth]{4-experiment-design/img/descent-phase.png}
    \end{align*}
    \caption{The emptying and sampling sequence-Descent Phase\label{fig:descent}}
\end{figure}

In the diagrams, 0 denotes closed/off and 1 denotes opened/on.

The general timeline of the experiment is as follow:

\textbf{Ascent Phase:}\\
$t_0$ – $t_1$
\begin{itemize}
    \item CAC valve shall be opened.
    \item AAC valves shall be opened.
    \item CAC tube shall be flushed.
    \item AAC sampling bags shall be flushed.
    \end{itemize}
$t_1$ – $t_2$
\begin{itemize}
    \item AAC valves shall be closed.
    \end{itemize}
$t_2$ – $t_3$
\begin{itemize}
    \item Sampling bags' control valve shall be closed.
    \item Sampling bag valve 1 shall be opened, allowing for air to enter the first bag.
    \end{itemize}
$t_3$ – $t_4$
\begin{itemize}
    \item Sampling bag valve 1 shall be closed
    \item Sampling bags' control valve shall be opened, allowing the system to flush. 
    \end{itemize}
$t_4$ - $t_{n-1}$
\begin{itemize}
    \item The above procedures shall repeat itself until the remaining nine bags have collected air samples for their assigned altitudes.
    \end{itemize}
$t_{n-1}$ – $t_n$
\begin{itemize}
    \item Sampling bag valve 10 shall be closed.
    \item Sampling bags' control valve shall be closed.
\end{itemize}


\textbf{\\Float Phase:}\\
No action taken other than continued telemetry. Air pump is off.\\
 
\textbf{Descent Phase:}

Note: Before sampling starts again, the system has to be flushed. 

$t_{n+1}$ – $t_{n+2}$
\begin{itemize}
    \item Sampling bags' control valve shall be closed.
    \item Sampling bag valve 11 shall be opened, allowing for air to enter the first bag.
\end{itemize}

$t_{n+2}$ – $t_{n+3}$
\begin{itemize}
    \item Sampling bags' control valve shall be closed and sampling bag valve 12 shall be opened.
\end{itemize}

In between, same procedure shall repeat itself until all the remaining bags have collected air samples for their assigned altitudes.

$t_{pre-landing}$
\begin{itemize}
    \item System shuts down:Sampling bag valve 16 shall be closed.
    \item Sampling bags' control valve shall be closed
    \item CAC valve shall be closed.
\end{itemize}


Note: The AAC system's air pump is only on during sampling into the air sampling bags and flushing of the system.

\raggedbottom