% Please add the following required packages to your document preamble:
% \usepackage{multirow}
\begin{table}[H]
\centering
\begin{tabular}{|l|l|l|l|l|}
\hline
 & \textbf{\begin{tabular}[c]{@{}l@{}}Minimum \\ Sampling Volume\end{tabular}} & \textbf{\begin{tabular}[c]{@{}l@{}}Sampling \\ Altitudes\end{tabular}} & \textbf{\begin{tabular}[c]{@{}l@{}}Ambient \\ Pressure\end{tabular}} & \textbf{\begin{tabular}[c]{@{}l@{}}Ambient \\ Temperature\end{tabular}} \\ \hline
\multirow{2}{*}{\textbf{Ascent Phase}} & 1.8 L & 18 km & 75.0 hPa & 216.7 K \\ \cline{2-5} 
 & 2.4 L & 21 km & 46.8 hPa & 217.6 K \\ \hline
\multirow{4}{*}{\textbf{Descent Phase}} & 1.7 L & 17.5 km & 81.2 hPa & 216.7 K \\ \cline{2-5} 
 & 1.3 L & 16 km & 102.9 hPa & 216.7 K \\ \cline{2-5} 
 & 1.0 L & 14 km & 141.0 hPa & 216.7 K \\ \cline{2-5} 
 & 0.7 L & 12 km & 193.3 hPa & 216.7 K \\ \hline
\end{tabular}
\caption{Sampling Altitudes as well as the Corresponding Ambient Pressures and Temperatures According to the 1976 US Standard Atmosphere and the Minimum Sampling Volume at Each Altitude to Obtain Enough Air to Perform a Proper Analysis (0.18 L at Sea Level), Appendix \ref{sec:appH}.}
\label{tab:minimum-volume}
\end{table}
