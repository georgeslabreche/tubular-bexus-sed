\pagebreak
\subsection{Power Design}

\subsubsection{Power System Requirements}
\begin{centering}
The Gondola provides 28.8 V or 13 Ah battery with a recommended maximum current draw of 1.8A \cite{BexusManual}. The experiment must run on external power from 4 hours before launch during the countdown phase and for the entire flight duration, lasting approximately 4 hours. As a factor of safety the experiment should be able to run for an additional 2 hours. 
\end{centering}

\begin{longtable}{|m{0.03\textwidth}| m{0.3\textwidth} |m{0.14\textwidth} |m{0.16\textwidth}|m{0.13\textwidth}| m{0.14\textwidth} |}
\hline
\textbf{ID}             & \textbf{Component}                                                   & \textbf{Voltage {[}V{]}} & \textbf{Current {[}mA{]}} & \textbf{Power {[}W{]}} & \textbf{Total {[}Wh{]}} \\ \hline
1                       & Arduino Due                                       & 12                                          & 30                                           & 0.36                                      & 36                                         \\ \hline
2                       & \hl{BTC Series PMDC Iron Core Brush Miniature Diaphragm Pump H084-11} \st{Micro Air Pump CMP30-3PW}                          & 24                                          & \hl{180} \st{300}                                          &\hl{4.32} \st{7.2}                                       & \hl{4.32} \st{7.2}                                        \\ \hline
3                       & Barometric Pressure Sensor MS5607-02BA03          & 3.3                                         & 1.4                                          & 0.00462                                   & 0.1                                        \\ \hline
4                       & Electromagnetically controlled valve              & 24                                          & 458                                          & 11                                        & 75                                         \\ \hline
5                       & Airflow sensor AWM40000 Series                    & 10                                          & 6                                            & 0.060                                     & 0.6                                        \\ \hline
6                       & Polyimide Thermofoil Heaters HK5161R78.4L12       & 28                                          & 357                                          & 10                                        & 100                                        \\ \hline
7                       & Polyimide Thermofoil Heaters HK5160R157L12        & 28                                          & 179                                          & 5                                         & 50                                         \\ \hline
8                       & Temperature sensor VSSOP-8, LM70, LM70CIMM-5/NOPB & 5.5                                         & 0.49                                         & 0.002695                                  & 0.054                                      \\ \hline
9                       & DC-DC step converter                              & 28                                          & 500 (output)                                 & 0.09                                      & 0.9                                        \\ \hline
10                      & HDC2010 Low Power Humidity Digital Sensors        & 3.3                                         & 0.0005                                       & 1.65$\times10^{-6}$                              & 16.5$\times10^{-6}$                               \\ \hline
\multicolumn{1}{|c|}{-} & \textbf{Total}                                  & \multicolumn{1}{c|}{-}                      & \multicolumn{1}{c|}{\hl{1671} \st{1791}}                    & \multicolumn{1}{c|}{\hl{41.42} \st{44.3}}                 & \hl{267} \st{270}                                        \\ \hline
\multicolumn{1}{|c|}{-} & \textbf{Available from gondola}                 & \multicolumn{1}{c|}{-}                      & \multicolumn{1}{c|}{-}                       & \multicolumn{1}{c|}{-}                    & 374                                        \\ \hline

\caption{Power Design Table}
\label{tab:power-design-table}
\end{longtable}
\raggedbottom

%   $16.5\times10^{blah}$

\subsubsection{Power System Control and Regulation}
The power on board will be split using three DC-DC converters. The pump and valves both have a high peak current. Therefore it was decided to use two DC-DC converters to step down from 28.8V to 24V giving the pump its own dedicated DC-DC converter. This is as the pump and valves both have a high peak current draw. The Arduino will also have its own DC-DC converter stepping the voltage down from 28.8V to 12V. It is thought using three DC-DC converters will provide the best compromise between efficiency, cost and heating.

\raggedbottom
