\pagebreak
\subsection{Power System}

\subsubsection{Power System Requirements}
\begin{centering}
The Gondola provided a 28.8 V, 374 Wh or 13 Ah battery with a recommended maximum continuous current draw of 1.8 A . However, more typical values which were given were 196 Wh or 7 Ah \cite{BexusManual}.The experiment should have been able to run on (gondola) battery for more than two hours before launch during the countdown phase and for the entire flight duration, lasting approximately four hours. As a factor of safety, in case of unexpected delays, the experiment was able to run for an additional four hours. Therefore the experiment could be able to run on (gondala) power for a total of 10 hours. For this reason, all the calculations were done using a 10 hour total time \cite{BexusManual}.
\end{centering}


\begin{longtable}{|m{0.03\textwidth}| m{0.3\textwidth} |m{0.14\textwidth} |m{0.16\textwidth}|m{0.13\textwidth}| m{0.14\textwidth} |}
\hline
\textbf{ID}             & \textbf{Component}                                                   & \textbf{Voltage {[}V{]}} & \textbf{Current {[}mA{]}} & \textbf{Power {[}W{]}} & \textbf{Total {[}Wh{]}} \\ \hline
1                       & Arduino Due                                       & 12                                          & 30                                           & 0.36                                      & 36                                         \\ \hline
2                       & Micro Air Pump CMP30-3PW                          & 24                                          & 300                                          & 7.2                                       & 7.2                                        \\ \hline
3                       & Barometric Pressure Sensor MS5607-02BA03          & 3.3                                         & 1.4                                          & 0.00462                                   & 0.1                                        \\ \hline
4                       & Electromagnetically controlled valve              & 24                                          & 458                                          & 11                                        & 75                                         \\ \hline
5                       & Airflow sensor AWM40000 Series                    & 10                                          & 6                                            & 0.060                                     & 0.6                                        \\ \hline
6                       & Polyimide Thermofoil Heaters HK5161R78.4L12       & 28                                          & 357                                          & 10                                        & 100                                        \\ \hline
7                       & Polyimide Thermofoil Heaters HK5160R157L12        & 28                                          & 179                                          & 5                                         & 50                                         \\ \hline
8                       & Temperature sensor VSSOP-8, LM70, LM70CIMM-5/NOPB & 5.5                                         & 0.49                                         & 0.002695                                  & 0.054                                      \\ \hline
9                       & DC-DC step converter                              & 28                                          & 500 (output)                                 & 0.09                                      & 0.9                                        \\ \hline
10                      & HDC2010 Low Power Humidity Digital Sensors        & 3.3                                         & 0.0005                                       & 1.65$\times10^{-6}$                              & 16.5$\times10^{-6}$                               \\ \hline
\multicolumn{1}{|c|}{-} & \textbf{Total}                                  & \multicolumn{1}{c|}{-}                      & \multicolumn{1}{c|}{1791}                    & \multicolumn{1}{c|}{44.3}                 & 270                                        \\ \hline
\multicolumn{1}{|c|}{-} & \textbf{Available from gondola}                 & \multicolumn{1}{c|}{-}                      & \multicolumn{1}{c|}{-}                       & \multicolumn{1}{c|}{-}                    & 374                                        \\ \hline

\caption{Power Design Table}
\label{tab:power-design-table}
\end{longtable}
\raggedbottom

%   $16.5\times10^{blah}$


The total power consumption 181 Wh, Table \ref{tab:power-design-table}, was within the limits of the available power. Other calculations for the average, peak, and minimum power values were 24 W, 38 W, and 16 W respectively. In addition the different expected current consumption for the average, peak, and minimum values were 0.64 A, 1.1 A, and 0.22 A respectively.

The 24 V DC-DC converters had 2.5 A output current and 60 W output power with the efficiency of 93\%. This fulfilled the peak requirements for both power and current. Moreover, the dissipated power and current across the DC-DCs were calculated as 12.69 Wh and 45 mA respectively and have been added to the total power budget. 



\raggedbottom
