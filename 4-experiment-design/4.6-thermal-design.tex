\pagebreak
\subsection{Thermal Design} \label{Thermal_section}

\subsubsection{Thermal Environment}
\begin{centering}
The experiment will experience a wide range of temperatures during the flight and it must be able to continue to operate despite these changes. As seen in Figure \ref{fig:temperature-profile} the coldest point of the flight will be between 15km and 20km where the air temperature can drop to $-80\degree$C and temperatures on the gondola have been recorded as low as $-40\degree$C during the float phase in the past \cite{BexusManual}. In addition launching from Kiruna in late October means the temperature on the ground could be as low as $-15\degree$C. As our lowest operating temperature component must be at a minimum of $0\degree$C this could mean heaters may need to be switched on while the experiment is still on the ground. 
\end{centering}


\begin{figure}[H]
    \begin{align*}
        \includegraphics[height=5cm]{4-experiment-design/img/temperature-profile.png}
    \end{align*}
    \caption{Diagram showing the temperature profile of the atmosphere \cite{jacob}}\label{fig:temperature-profile}
\end{figure}

\subsubsection{Overall Design}
\begin{centering}
To protect the components against the cold two heaters will be included. One of these will be placed close to the Arduino, which is anticipated to be able to keep the controller area warm enough, and the second will be placed close to the valves to ensure they continue to operate as expected. To control these heaters two temperature sensors will also be onboard in similar locations. If the reading from one of the temperature sensors is lower than the predefined value then the heater will turn on. 
\end{centering}


\begin{centering}
In addition to using electrical heaters the experiment will also be thermally insulated. The Styrofoam casing which will be used to protect the CAC from impact forces will also serve as insulation. It is also planned to add additional insulation around the components.  
\end{centering}

\begin{centering}
It is also important to consider heating from the sun which could raise the temperature of the experiment considerably. Sufficient insulation should be included to ensure the inside of the box stays within the operating temperature range. This will be investigated at a later date when a full CFD thermal analysis is completed.
\end{centering}
\bigskip

\pagebreak

\begin{longtable}{|m{1cm}|m{3.5cm}|m{1cm}|m{1cm}|m{1cm}|m{1cm}|}
\hline
\multirow{2}{*}{\textbf{ID}} & \multirow{2}{*}{\textbf{Components}}                                 & \multicolumn{2}{l|}{\textbf{Operating T (°C)}} & \multicolumn{2}{l|}{\textbf{Survivable (°C)}} \\ \cline{3-6} 
                             &                                                                      & Min.                   & Max.                  & Min.                  & Max.                  \\ \hline
1                            & Arduino Due                                                          & -40                    & 85                    & -60                   & 150                   \\ \hline
2                            & Arduino Ethernet Shield 2                                            & -40                    & 85                    & -65                   & 150                   \\ \hline
3                            & Barometric Pressure Sensor MS5607-02BA03                             & -40                    & 85                    & -40                   & 125                   \\ \hline
4                            & Airflow sensor AWM40000 Series                                       & -40                    & 125                   & -40                   & 125                   \\ \hline
6                            & Temperature sensor VSSOP-8, LM70, LM70CIMM-5/NOPB & -55                    & 150                   & -65                   & 150                   \\ \hline
7                            & HDC2010 Low Power Humidity Digital Sensors                           & -40                    & 85                    & -40                   & 125                   \\ \hline
8                            & Industrial temperature microSD XCUHS-I 8GB                           & -40                    & 85                    & -40                   & 85                    \\ \hline
9                            & DC-DC step converter                                                 & -40                       & 85                       & -50                      & 125                      \\ \hline
10                           & Electromagnetically controlled valve                                 &  -43                      & 85              & -43                      & 100                      \\ \hline
11                           & Micro Air Pump CMP30-3PW                                                             & 0                       & 50                      & 0                      & 50                      \\ \hline

\caption{Thermal Table}
\label{tab:thermal-table}
\end{longtable}
\raggedbottom

\raggedbottom

\subsubsection{Internal Temperature}

As the current experiment model stands, an enclosed partition has been reserved in the lower front-right corner of the AAC section of the TUBULAR experiment. This partition will house all of the electronic components not required to be situated in specified locations throughout the experiment setting, such as some of the sensors. 

Inside this partition will be three separate insulated boxes. The first of these is the Electronics box which will occupy a space measuring 20 cm in length, 10 cm in width, and 20 cm in height, and will be insulated from inside to outside with polyethylene, polystyrene, and finally aluminum. The infrastructure within this enclosure shall be such that the electronics are fitted together to in turn be bound within another box of insulated material (likely made from PVC). 

Above the electronics box will be the second internal box, the pump box, where the pump will be situated and the third internal box, the valve centre, where the valves will be situated. The valves will be routed together by several tubes to form a compact structure aiding the thermal control over this area. 

One heater will be included in the electronics box with its main priority being keeping the Arduino Due board within operational temperatures whilst delivering the remaining heat to peripheral electronics. The other heater will be fitted between the pump and the entrances to the valve center and will keep its lateral neighbors within their respective operational temperature ranges.

The pump has the most critical temperature range as it is the only component that cannot operate below freezing temperatures. It's data sheet states it must always be no colder than $5\degree$C, or the EPDM diaphragm may not be able to expand and contract sufficiently to maintain the desired airflow of 8L/min. However as this pump has been used successfully on flights before, \cite{LISA}, tests will be conducted on the pump to find its true performance at lower temperatures. The valves are also crucial to the experiment's function, as they enable each and every sampling bag onboard to be used. For this reason, while the valves can operate down to $-43\degree$C, it is desirable to be keep them above this limit whenever in use.

Given the thermal conductivity of EPDM ($0.2 W/(m.K)$), the pump's diaphragm experiences a minimal temperature drop across itself when one side is subject to the heater's temperature while the other end of the pump is leveled with the ambient temperature. Through calculation it was found that at an ambient temperature of $-50\degree$C and a heater temperature at $10\degree$C, the center of the pump, the far end of the diaphragm, measured $9.35\degree$C. Reassessing the equation yielded the most energy-conserving temperature that would keep the pump's diaphragm above the minimum requirement to be $7\degree$C. This held whether the temperature was as it could be on ground level ($-10\degree$C), or in floating phase ($-50\degree$C).

Symbols used are the following:

\begin{itemize}
    \item $T_{inlet}$ &= Temperature at the pump inlet face
    \item $T_{heated}$ &= Temperature at the pump's diaphragm face (directly heated)
    \item $L_{inlet}$ &= Length of the pump inlet valve(s)
    \item $L_{phragm}$ &= Length of the pump's diaphragm
    \item $k_{PPS}$ &= Thermal conductivity of the pump's inlet valve(s) made of PPS
    \item $k_{EPDM}$ &= Thermal conductivity of the pump's diaphragm made of EPDM
    \item $k_{air}$ &= Thermal conductivity of the air entering the pump
    \item $k_{phragm}$ &=  Averaged thermal conductivity of the pump's diaphragm (Assumed 4 parts air to 1 part EPDM)
    \item $T_{med}$ &= Temperature halfway across the pump (start of the diaphragm)
\end{itemize}




 \begin{align*}
    T_{inlet} &= 263 K\\
    T_{heated} &= 283 K\\
    L_{inlet} &= 0.038 m\\
    L_{phragm} &= 0.038 m\\
    k_{PPS} &= 2\frac{W}{m\cdot K}\\
    k_{EPDM} &= 0.2\frac{W}{m\cdot K}\\
    k_{air} &= 0.024\frac{W}{m\cdot K}\\
    k_{phragm} &=  \frac{(k_{EPDM}+4\cdotk_{air})}{5} = 0.06\frac{W}{m\cdotK}\\
    T_{med} &= \frac{( k_{phragm}\cdot   L_{inlet}\cdot T_{inlet}) + (k_{PPS}\cdot L_{phragm}\cdot T_{heated}) }{ (k_{phragm}\cdot L_{inlet}) + (k_{PPS} \cdot L_{phragm})} \\
    T_{med} &= \frac{ (0.06\cdot  0.038\cdot 263) + (2\cdot 0.038\cdot 283)}{(0.06\cdot 0.038) + (2\cdot 0.038)} \\
    T_{med} &= 282.35 K \longrightarrow 9.35\degree C
 \end{align*}

Reworking the equation to fit the inlet face at the true ambient temperature of -50$\degree$C and the reduced heater temperature of 7$\degree$C gives the following:

% nice :) Also remove the * and it will number the equations. Also also to start a new line just write \\ at the end and it will make the next thing appear on a new line. Also also also using &= aligns all the equals signs and makes it look nice :D I leave these tips here so delete when you don't need them :)


 \begin{align*}
     T_{inlet} &= 223 K\\
    T_{heated} &= 280 K\\
    T_{med} &= \frac{ (k_{phragm}\cdot L_{inlet}\cdot T_{inlet}) + (k_{PPS}\cdot L_{phragm}\cdot T_{heated}) }{ (k_{phragm}\cdot L_{inlet}) + (k_{PPS}\cdot L_{phragm})}\\
    T_{med} &= \frac{ (0.06\cdot 0.038\cdot 223) + (2\cdot 0.038\cdot 280)}{(0.06\cdot 0.038) + (2\cdot 0.038)}\\
    T_{med} &= 278.28 K \longrightarrow 5.28\degree C
 \end{align*}

\newpage
\subsubsection{Box Analysis}

For the box temperature to be properly estimated, the other two divisions within the component enclosure need to be taken into account. The manifold comprising the valve center is made of a compound of nylon (assumed to be nylon-6) and kynar. It has also been assumed they are used equally in mass and distribution at this stage of the thermal estimation.

Symbols used are the following:

\begin{itemize}
    \item $T_{outlet}$ &= Temperature at the manifold's outlet(s)
    \item $T_{heated}$ &= Temperature of the pump at its diaphragm face
    \item $L_{outhalf}$ &= Length of the outer half of the manifold
    \item $L_{inhalf}$ &= Length of the inner half of the manifold
    \item $k_{Nylon-6}$ &= Thermal conductivity of Nylon-6
    \item $k_{Kynar}$ &= Thermal conductivity of Kynar
    \item $k_{manifold}$ &= Averaged thermal conductivity of manifold
\end{itemize}


 \begin{align*}
    T_{outlet} &= 263 K\\
    T_{heated} &= 283 K\\
    L_{outhalf}& = 0.038 m\\
    L_{inhalf} &= 0.038 m\\
    k_{Nylon-6} &= 0.24\frac{W}{m\cdot K}\\
    k_{Kynar} &= 0.162\frac{W}{m\cdot K}\\
    k_{manifold} &=  \frac{k_{Nylon-6}+k_{Kynar}}{2} = 0.201\frac{W}{m\cdot K}\\
    T_{med} &= \frac{ (k_{manifold}\cdot L_{inhalf}\cdot T_{heated}) + (k_{manifold}\cdot L_{outhalf}\cdot T_{outlet}) }{ (k_{manifold}\cdot L_{inhalf}) + (k_{manifold}\cdot L_{outhalf})}\\
    T_{med} &= \frac{ (0.201\cdot 0.040\cdot 223) + (0.201\cdot 0.040\cdot 280)}{(0.201\cdot 0.040) + (0.201\cdot 0.040)}\\ 
    T_{med} &= 251.59 K \longrightarrow -21.41\degree C
 \end{align*}
 
 
\newpage
Below, in Figure \ref{fig:enclosurebox} is a theoretical general arrangement of the components within the enclosure, seen from the front of the experiment inward.
 
 \begin{figure}[H]
    \begin{align*}
        \includegraphics[height=8cm]{4-experiment-design/img/bexus-thermal-first.png}
    \end{align*}
    \caption{Diagram showing a general layout of the AAC components}
    \label{fig:enclosurebox}
\end{figure}
 
 A rough estimation of temperature gradients can thus be shown in Figure \ref{fig:boxanalysis}:
 
 \begin{figure}[H]
    \begin{align*}
        \includegraphics[height=8cm]{4-experiment-design/img/bexus-thermal-second.png}
    \end{align*}
    \caption{Diagram of the previous layout now showing temperature contours}
    \label{fig:boxanalysis}
\end{figure}

