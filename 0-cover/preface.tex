\section*{PREFACE} \markboth{}{}
\addcontentsline{toc}{section}{PREFACE}

The Rocket and Balloon Experiments for University Students (REXUS/BEXUS) programme is realized under a bilateral Agency Agreement between the German Aerospace Center (DLR) and the Swedish National Space Agency (SNSA). The Swedish share of the
payload has been made available to students from other European countries through a collaboration with the European Space Agency (ESA).

EuroLaunch, a cooperation between the Esrange Space Center of SSC and the Mobile Rocket Base (MORABA) of DLR, is responsible for the campaign management and operations of the launch vehicles. Experts from DLR, SSC, ZARM, and ESA provide
technical support to the student teams throughout the project.

The Student Experiment Documentation (SED) is a continuously updating document regarding the BEXUS student experiment TUBULAR - Alternative to AirCore for Atmospheric Greenhouse Gas Sampling and will undergo reviews during the preliminary design review, the critical design review, the integration progress review, and final experiment report.

The TUBULAR Team consists of a diverse and inter-disciplinary group of students from Luleå University of Technology's Masters programme in Atmospheric Studies, Space Engineering, and Spacecraft Design. The idea for the proposed experiment stems from concerns over the realities of climate change as a result of human activity coupled with the complexity and limitations in obtaining greenhouse gas profile data to support climate change research.

Based above the Arctic circle in Kiruna, Sweden, the TUBULAR Team is exposed to Arctic science research with which it has collaborated in order to produce research detailing the air sampling methodology, measurements, analysis, and findings.

