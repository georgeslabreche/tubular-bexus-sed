


Carbon dioxide (CO$_{2}$), methane (CH$_{4}$), and carbon monoxide (CO) are three main greenhouse gases emitted by human activities. Developing a better understanding of their contribution to greenhouse effects requires more accessible, flexible, and scalable air sampling mechanisms. A balloon flight is the most cost-effective mechanism to obtain a vertical air profile through continuous sampling between the upper troposphere and the lower stratosphere. However, recovery time constraints due to gas mixture concerns geographically restrict the sampling near existing research centers where analysis of the recovered samples can take place. The TUBULAR experiment is a technology demonstrator for atmospheric research supporting an air sampling mechanism that would offer climate change researchers access to remote areas by minimizing the effect of gas mixtures within the collected samples so that recovery time is no longer a constraint. The experiment will include a secondary sampling mechanism that will serve as reference against which the proposed sampling mechanism can be validated.

%\footnote{The main reason for changing the type of gas that will be detected from N$_2$O to CO is that the model of analyzer used is only able to detect CO$_{2}$, CH$_{4}$ and CO.\label{fn:ChangeN2OtoCO}}