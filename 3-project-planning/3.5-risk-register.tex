\pagebreak
\subsection{Risk Register}
\textbf{Risk ID}
\begin{enumerate}[label={}]
    \item TC – technical/implementation 
    \item MS – mission (operational performance) 
    \item SF – safety 
    \item VE – vehicle 
    \item PE – personnel 
    \item EN – environmental 
    \item OR - Outreach
    \item BG - Budget
\end{enumerate}

Adapt these to the experiment and add other categories. 
Consider risks to the experiment, to the vehicle and to personnel. 

\textbf{Probability (P)}
\begin{enumerate}[label=\Alph*]
    \item Minimum – Almost impossible to occur 
    \item Low – Small chance to occur 
    \item Medium – Reasonable chance to occur 
    \item High – Quite likely to occur 
    \item Maximum – Certain to occur, maybe more than once
\end{enumerate}

\textbf{Severity (S)}
\begin{enumerate}
    \item Negligible – Minimal or no impact 
    \item Significant – Leads to reduced experiment performance 
    \item Major – Leads to failure of subsystem or loss of flight data 
    \item Critical – Leads to experiment failure or creates minor health hazards 
    \item Catastrophic – Leads to termination of the BEXUS programme, damage to the vehicle or injury to personnel 
\end{enumerate}

The rankings for probability (P) and severity (S) are combined to assess the overall risk classification, ranging from very low to very high and being coloured green, yellow, orange or red according to the SED guidelines.

\begin{landscape}
\begin{longtable}{|m{0.09\textwidth}| m{0.51\textwidth} |m{0.03\textwidth} |m{0.03\textwidth}|m{0.11\textwidth}| m{0.65\textwidth}|}

\hline
\textbf{ID} & \textbf{Risk (\& consequence if)} & \textbf{P} & \textbf{S} & \textbf{P * S} & \textbf{Action} \\ \hline
TC10 & Software fails to store data & B & 4 & \cellcolor[HTML]{FCFF2F}Low & Acceptable Risk: Extensive testing will be done \\ \hline
TC20 & Failure of several sensors & B & 3 & \cellcolor[HTML]{FCFF2F}Low & Acceptable Risk: Thermal test to approve the functionality of the experiment. \\ \hline
TC30 & Critical component is destroyed in testing & B & 1 & \cellcolor[HTML]{FCFF2F}Low & Acceptable risk: Spare components can be ordered but for expensive ones, they will be ordered and tested early in the project in case we need to order more. \\ \hline
TC40 & Electrical connections dislodges or short circuits because of vibration or shock & B & 3 & \cellcolor[HTML]{FCFF2F}Low & Unacceptable risk. Careful soldering and extensive testing will be applied. \\ \hline
TC50 & Experiment electronics fail due to long exposure to cold or warm temperatures & B & 3 & \cellcolor[HTML]{FCFF2F}Low & Unacceptable Risk: Thermomechanical and thermoelectrical solutions will be simulated and tested in detail to help prevent this from happening. \\ \hline
TC60 & Software and electrical fail to control heaters causing temperature to drop or rise below or above operational range & B & 2 & \cellcolor[HTML]{34FF34}Very Low & Unacceptable risk: Tests will be performed prior to the flight to detect and minimize the risk of occurrence.The system will be monitored during flight and handled manually if necessary. \\ \hline
TC70 & Software fails to enter safe mode (may result in loss of data) & B & 3 & \cellcolor[HTML]{FCFF2F}Low & Acceptable Risk: Extensive testing will be done. \\ \hline
TC80 & On-board memory will be full (flight time longer than expected) & A & 2 & \cellcolor[HTML]{34FF34}Very Low & Acceptable Risk: The experiment shall go through testing and analysis to guarantee the onboard memory size is sufficient.\\ \hline
TC90 & Connection loss with ground station & A & 4 & \cellcolor[HTML]{34FF34}Very Low & Acceptable Risk: Experiment will be designed to operate autonomously. \\ \hline
TC100 & Software fails to control valves autonomously & B & 4 & \cellcolor[HTML]{FCFF2F}Low & Acceptable Risk: Extensive testing will be done. Telecommand will also be used to manually control the valves. \\ \hline
TC110 & Software fails to change modes autonomously & B & 4 & \cellcolor[HTML]{FCFF2F}Low & Acceptable Risk: Extensive testing will be done. Telecommand will also be used to manually change experiment modes. \\ \hline
TC120 & Complete software failure & B & 4 & \cellcolor[HTML]{FCFF2F}Low & Acceptable Risk: Watchdog timer will be applied to reset software if it freezes. \\ \hline
TC130 & Failure of fast opening system & B & 2 & \cellcolor[HTML]{34FF34}Very Low & Acceptable risk: the box could also be opened but a little bit slower. \\ \hline
TC140 & The gas analyzer isn't properly calibrated and returns inaccurate results & B & 4 & \cellcolor[HTML]{FCFF2F}Low & Acceptable risk: Calibrate the gas analyzer before use.\\ \hline
MS10 & Down link connection is lost prematurely & B & 2 & \cellcolor[HTML]{34FF34}Very Low & Acceptable Risk: Data will also be saved on SD card. \\ \hline
MS20 & Condensation on experiment PCBs which could causes short circuits & A & 3 & \cellcolor[HTML]{34FF34}Very Low & Acceptable risk: Circuit box will be sealed to prevent condensation. \\ \hline
MS30 & Temperature sensitive components that are essential to full the mission objective might be below their operating temperature. & C & 3 & \cellcolor[HTML]{FCFF2F}Low & Acceptable Risk: Safe mode to prevent the components to operate out of its operating temperature range. \\ \hline
MS40 & Experiment lands in water causing electronics failure & B & 1 & \cellcolor[HTML]{34FF34}Very Low & Acceptable risk: Check if SD card needs waterproof shell or is waterproof in itself. Also, all the necessary data will be downloaded during the flight. \\ \hline
MS50 & Interference from other experiments and/or balloon & A & 4 & \cellcolor[HTML]{34FF34}Very Low & Acceptable risk: no action. \\ \hline
MS60 & Balloon power failure & C & 1 & \cellcolor[HTML]{34FF34}Very Low & Acceptable risk: Valves default state is closed so if all power is lost valves will automatically close preserving all samples collected up until that point. \\ \hline
MS70 & Bags disconnect & C & 3 & \cellcolor[HTML]{FCFF2F}Low & Acceptable Risk: The affected bags could not collect samples. A proper fixing of the flanges must be double checked.
\\ \hline
MS71 & Bags puncture & B & 3 & \cellcolor[HTML]{FCFF2F}Low & Acceptable Risk: The affected bags could not collect samples. Proper protection will be placed in order to avoid puncture from external elements. \\ \hline
MS72 & Bags' hold time is typically 48h & C & 3 & \cellcolor[HTML]{FCFF2F}Low & Acceptable risk: Validation studies can demonstrate longer stability.  \\ \hline
MS80 & Pump failure & C & 4 & \cellcolor[HTML]{ffae42}Medium & Unacceptable risk: The bags would not be filled and thus the AAC system would fail. The pump will be properly chosen based on past research and extensively tested before the flight. \\ \hline
MS90 & Intake pipe blocked by external element & C & 3 & \cellcolor[HTML]{FCFF2F}Low & Unacceptable Risk: The bags would not be filled and thus the AAC system would fail. An air filter will be placed in both intake and outlet of the pipe to prevent this. \\ \hline
VE10 & SD-card is destroyed at impact & B & 3 & \cellcolor[HTML]{FCFF2F}Low & Acceptable Risk: All data will be transmitted to the ground. \\ \hline
VE20 & Gondola Fixing Interface & B & 4 & \cellcolor[HTML]{FCFF2F}Low & Unacceptable Risk: The experiment box could detach from the gondola’s rails and the two boxes could detach one from the other. Proper fixing has been designed to prevent it. \\ \hline
VE30 & Structure damage due to bad landing & B & 3 & \cellcolor[HTML]{FCFF2F}Low & Acceptable Risk: Landing directly on a hard element could break the structure or the protective walls. Consistent design implemented to prevent it. \\ \hline
VE40 & Hard landing damages the CAC equipment & C & 3 & \cellcolor[HTML]{FCFF2F}Low & Acceptable risk:  Proper  protection will be placed in order to avoid damage from hard landing. \\ \hline
VE50 & Hard landing damages the AAC equipment & C & 3 & \cellcolor[HTML]{FCFF2F}Low & Acceptable risk:  Proper  protection will be placed in order to avoid puncture from hard landing. \\ \hline
EN10 & Vibrations & C & 1 & \cellcolor[HTML]{34FF34}Very Low & Acceptable risk: Vibrations do not affect the sampled air. \\ \hline
EN20 & The air samples must be protected from direct sunlight and stored above 0 \degree C to prevent condensation & D & 4 & \cellcolor[HTML]{FF0800}High risk & Unacceptable risk: Further test regarding insulation performance and humidity levels in the bags will be done.  \\ \hline 
PE10 & The Project Manager is no longer available to manage the project. & E & 1 & \cellcolor[HTML]{FCFF2F}Low & Acceptable risk: The Deputy Project Manager will take over as Project Manager. \\ \hline 
PE20 & Team members from the same division are unavailable during the same period over the summer. & C & 4 & \cellcolor[HTML]{ffae42}Medium risk & Unacceptable risk: Summer travel schedules to be coordinated among team members and approved by Project Manager. \\ \hline 
% & & & & &\\ \hline

%\end{tabular}
\caption{Risk Register}
\label{tab:risk-register}
\end{longtable}
\raggedbottom
\end{landscape}