\subsection{Resources}

\subsubsection{Manpower}
The TUBULAR Team is categorized into divisions as summarized in Table \ref{tab:divisions-members}:

\begin{table}[H]
\centering
\resizebox{\textwidth}{!}{%
\begin{tabular}{|l|l|l|l|l|}
\hline
\textbf{Management} & \textbf{Scientific}    & \textbf{Mechanical} & \textbf{Electrical} & \textbf{Software}          \\ \hline
Georges Labrèche*    & Kyriaki Blazaki*        & Pau Molas Roca*      & Hamad Siddiqi*       & Muhammad Ansyar Rafi Putra* \\ \hline
                    & Nuria Agues Paszkowsky & Jordi Coll Ortega   & Natalie Lawton      & Gustav Dyrssen             \\ \hline
                    &                        &                     & Ivan Zankov         &                            \\ \hline
\end{tabular}%
}
\caption{Project Divisions and Members (Asterisks Denote Division Leaders)}
\label{tab:divisions-members}
\end{table}
\raggedbottom

The experience of TUBULAR Team members are listed in Table \ref{tab:team-member-experience}:

% Please add the following required packages to your document preamble:
% \usepackage{graphicx}
\begin{table}[H]
\centering
\begin{tabular}{|l|m{11cm}|}
\hline
\textbf{Team Member} & \textbf{Project Related Experience} \\ \hline
Natalie Lawton & MSc in Spacecraft Design (2nd Year). \\& MEng in Aerospace Engineering.\\& Previous experience in UAV avionic systems and emissions measurement techniques. \\ \hline
Nuria Agues Paszkowsky & MSc in Earth Atmosphere and the Solar System (2nd Year). \\& BSc in Aerospace Engineering.\\ \hline
Kyriaki Blazaki & MSc in Earth Atmosphere and the Solar System (2nd Year). \\& BSc in Physics. \\ \hline
Emily Chen & MSc in Space Engineering (5th Year). \\ \hline
Jordi Coll Ortega & MSc in Spacecraft Design (2nd Year). \\& BSc in Aerospace Vehicle Engineering. \\ \hline
Gustav Dyrssen &  MSc in Space Engineering (5th Year).\\ \hline
Erik Fagerström & MSc in Space Engineering (5th Year). \\ \hline
Georges L. J. Labrèche & MSc in Spacecraft Design (2nd Year). \\& BSc in Software Engineering.\\ \hline
Muhammad Ansyar Rafi Putra & MSc in Spacecraft Design (2nd Year). \\& BSc in Aerospace Engineering. \\ \hline
Pau Molas Roca & MSc in Spacecraft Design (2nd Year). \\& BSc in Aerospace Technology Engineering, Mechanical experience. \\ \hline
Emil Nordqvist & MSc in Space Engineering (5th Year). \\ \hline
Hamad Siddiqi & MSc Satellite Engineering (4th Year) \\&  BSc in Electrical Engineering.\\& Experience in telecommunication industry and electronics.  \\ \hline
Ivan Zankov & MSc in Spacecraft Design (2nd Year). \\& BEng in Mechanical Engineering.\\ \hline
\end{tabular}
\caption{Project Related Experience of Team Members.}
\label{tab:team-member-experience}
\end{table}
\raggedbottom

The initial projected effort to be contributed by each team member was averaged at 1.5 hour per person per day corresponding to a team total of 15 hours per day. Since then, 3 new members have been included in the team thus increasing the projected daily effort to 19.5 hours per day. During the period leading up to the launch it is expected that the effort contributed will be double to 3 hours per person per day. The period of these different effort capacities are listed in Table \ref{tab:daily-team-effort-per-period}:

\begin{table}[H]
\centering
\begin{tabular}{|c|c|c|}
\hline
\textbf{From} & \textbf{To} & \textbf{Capacity (hours/day)} \\ \hline
08/01/2018 & 18/03/2018 & 15 \\ \hline
19/03/2018 & 08/04/2018 & 16.5 \\ \hline
09/04/2018 & 09/05/2018 & 18 \\ \hline
10/05/2018 & 15/08/2018 & 19.5 \\ \hline
15/08/2018 & 22/10/2018 & 39 \\ \hline
23/10/2018 & 31/01/2019 & 19.5 \\ \hline
\end{tabular}
\caption{Projected Daily Team Effort per Period.}
\label{tab:daily-team-effort-per-period}
\end{table}

Taking into account all team members and the mid-project changes in team size, the efforts/capacity projected to be allocated to each stages of the project during 2018 are summarized in Table \ref{tab:effort-allocation-stages}:

\begin{table}[H]
\centering
\begin{tabular}{l|l|l|}
\hline
\multicolumn{1}{|l|}{\textbf{Stage}} & \textbf{Duration (days)} & \textbf{Effort (hours)} \\ \hline
\multicolumn{1}{|l|}{Preliminary Design} & 28 & 420 \\ \hline
\multicolumn{1}{|l|}{Critical Design} & 70 & 1050 \\ \hline
\multicolumn{1}{|l|}{Experiment Building and Testing} & 45 & 675 \\ \hline
\multicolumn{1}{|l|}{Final Experiment Preparations} & 45 & 675 \\ \hline
\multicolumn{1}{|l|}{Launch Campaign} & 10 & 150 \\ \hline
\multicolumn{1}{|l|}{Data Analysis and Reporting} & 66 & 990 \\ \hline
\multicolumn{1}{r|}{\textbf{Total:}} & \textbf{264} & \textbf{3960} \\ \cline{2-3} 
\end{tabular}
\caption{Project Effort Allocation per Project Stages}
\label{tab:effort-allocation-stages}
\end{table}
\raggedbottom

All TUBULAR Team members are based in Kiruna, Sweden, just \SI{40}{\kilo\meter} from Esrange Space Center. Furthermore, all team members are enrolled in LTU Master programmes in Kiruna and thus expected to remain in LTU during the entire project period. Special attention will have to be made for planning tasks during the summer period where many team members are expected to travel abroad. A timeline of team member availability  until January 2019 is available in Appendix \ref{sec:appD}. A significant risk can be observed during the summer months from June to August where most members will only be partially available and some completely unavailable. As such, team member availability and work commitments over the summer have been negotiated across team members in order to guarantee that at least one member per division is present in Kiruna over the Summer with the exception of the Software Division which can work remotely. Furthermore, the Project Manager role will have to be assigned to the Deputy Project Manager due to an extended full time unavailability after the CDR.

As part of their respective Master programmes, most TUBULAR Team members are enrolled in a project course at LTU. The TUBULAR project acts as the course's project for most team members from which they will obtain ECTS credits. This course is supervised by Dr. Thomas Kuhn, Associate Professor at LTU.

\pagebreak
\subsubsection{Budget}
\label{sec:3.2.2}
The experiment has a total mass of \SI{24}{\kilo\gram} at a cost of 33,211 EUR. An error margin is included in the budget corresponding to 10\% of the total costs of components to be purchased. A complete budget is available in Appendix \ref{sec:appO} and a detailed component mass and cost breakdown is available in Section \ref{sec:experiment-components} Experiment Components. This breakdown does not include spare components accounted for in the total costs. Dimensions and mass of the experiment are summarized in Table \ref{table:experiment-summary} in Section \ref{sec:mechanical-design} and Table \ref{tab:dim-mass-tab} in Section \ref{sec:dim-mass}. A contingency fund of 900 EUR is allocated for unseen events such as component failures. Component loan and donations from sponsors account for 85\% of the project's total cost. LTU and SNSB funding accounts for the remaining 15\%. 

%
\begin{table}[H]
\centering
\begin{tabular}{|c|d{2}|d{2}|}%{D{.}{.}{1}}
\hline

\textbf{Category} & \textbf{Total Mass [g]} & \textbf{Total Price [EUR]} \\ \hline
Structure & 11,900.64 & 851.57 \\ \hline
Electronics Box & 521.50 & 1,618.12 \\ \hline
Cables and Sensors & 1,176.62 & 696.08 \\ \hline
CAC & 6,957.00 & 22,948.01 \\ \hline
AAC & 3,718.00 & 4,264.32 \\ \hline
Tools & - & 1,745.00 \\ \hline
Travel & - & 500.00 \\ \hline
Contingency & - & 900.77 \\ \hline
{\textbf{Total without Error Margin}} & \textbf{24,273.75} & \textbf{32,623.11} \\ \hline
Shipping Costs and Error Margin & 2,427.38 & 588.13 \\ \hline
{\textbf{Total with Error Margin}} & \textbf{26,701.13} & \textbf{33,211.24} \\ \hline
\end{tabular}
\caption{Mass and Cost Budget.}
\label{table:mass-and-cost-budget}
\end{table}

\raggedbottom

The project benefits from component donations from Restek, SMC Pneumatics, Teknolab Sorbent, KNF, Eurocircuits, and Lagers Masking Consulting as well as component loans from FMI. Furthermore, discounts were offered by Teknolab Sorbent and Bosch Rexroth. Euro value allocation of these sponsorships are presented in Table \ref{table:sponsroship-allocation}.

\begin{table}[H]
\centering
\begin{tabular}{l|m{0.12\textwidth}|l|r|r|r|c}
\hline
\multicolumn{1}{|l|}{\textbf{Sponsor}} & \multicolumn{1}{|c|}{\textbf{Type}} & \multicolumn{1}{c|}{\textbf{Value}} & \multicolumn{1}{c|}{\textbf{Allocated}} & \multicolumn{1}{c|}{\textbf{Unallocated}} & \multicolumn{1}{c|}{\textbf{\% Allocation}} & \multicolumn{1}{c|}{\textbf{Status}} \\ \hline
\multicolumn{1}{|l|}{LTU} & Funds & 2,500.00 & 1,880.33 & 619.67 & 75 & \multicolumn{1}{c|}{Received} \\ \hline
\multicolumn{1}{|l|}{SNSB} & Funds & 2,909.80 & 2,475.60 & 434.20 & 85 & \multicolumn{1}{c|}{Received} \\ \hline
\multicolumn{1}{|l|}{FMI} & Component loan & 22,500.00 & 22,500.00 & 0.00 & 100 & \multicolumn{1}{c|}{Confirmed} \\ \hline
\multicolumn{1}{|l|}{Restek} & Component donation & 1040.00 & 1040.00 & 0.00 & 100 & \multicolumn{1}{c|}{Received} \\ \hline
\multicolumn{1}{|l|}{Teknolab} & Component donation & 200.00 & 200.00 & 0.00 & 100 & \multicolumn{1}{c|}{Received} \\ \hline
\multicolumn{1}{|l|}{SMC} & Component donation & 860.00 & 860.00 & 0.00 & 100 & \multicolumn{1}{c|}{Confirmed} \\ \hline
\multicolumn{1}{|l|}{Lagers Maskin} & Component donation & 300.00 & 300.00 & 0.00 & 100 & \multicolumn{1}{c|}{Received} \\ \hline
\multicolumn{1}{|l|}{Swagelok} & Component donation & 1971.65 & 1971.65 & 0.00 & 100 & \multicolumn{1}{c|}{Confirmed} \\ \hline
 & \multicolumn{1}{l|}{\textbf{Total}} & \textbf{32,281.44} & \textbf{31,227.58} & \textbf{1053.86} & \textbf{95} & \multicolumn{1}{l}{} \\ \cline{2-6}
\end{tabular}
\caption{Allocation of Sponsorship Funds and Component Donation Values. Amounts in EUR.}
\label{table:sponsroship-allocation}
\end{table}


\subsubsection{External Support}

Partnership with FMI, and IRF will provide the team with technical guidance in implementing the sampling system. FMI’s experience in implementing past CAC sample systems provide invaluable lessons learned towards conceptualizing, designing, and implementing the proposed AAC sampling system.

FMI is a key partner in the TUBULAR project, its scientific experts will advise and support the TUBULAR project by sharing knowledge, experience, and granting accessibility of equipment. As per the agreement shown in Appendix \ref{sec:appG}, FMI will provide the TUBULAR Team with the AirCore stainless tube component of the CAC subsystem as well as the post-flight gas analyzer. This arrangement requires careful considerations on the placement of the experiment in order to minimize hardware damage risks. These contributions result in significant cost savings regarding equipment and component procurement.

Daily access to LTU's Space Campus in Kiruna exposes the team to scientific mentorship and expert guidance from both professors and researchers involved in the study of greenhouse gases and climate change. Dr Uwe Raffalski, IRF, Associate professor (Docent) is one of many researchers involved in climate study who is mentoring the team.