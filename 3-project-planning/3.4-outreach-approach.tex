
\subsection{Outreach Approach}

The experiment as well as the REXUS/BEXUS programme and its partners has been be promoted through the following activities:

\begin{itemize}
\item Published research papers co-authored with FMI, detailing the sampling methodology, measurement result, analysis, and findings \cite{esapac} \cite{labreche2020}.
\item Collected data licensed as open data to be freely available to everyone to use and republish as they wish, without restrictions from copyright, patents or other mechanisms of control.
\item A website to summarize the experiment and provide regular updates. Backend web analytics included to gauge interest on the project through number of visitors and their origins (See Appendix \ref{sec:appE}).
\item Dedicated Facebook page used as publicly accessible logbook detailing challenges, progress, and status of the project. Open for comments and questions (See Figure \ref{fig:outreach-facebook} in Appendix \ref{sec:appE}).
\item An Instagram account for short and frequent image and video updates (See Figures \ref{fig:outreach-instagram} in Appendix \ref{sec:appE}).
\item GitHub account to host all project software code under free and open source license (See Figure \ref{fig:outreach-github} in Appendix \ref{sec:appE}). Other REXUS/BEXUS teams were invited to host their code in this account.
\item\enquote{Show and Tell} trips to local high schools and universities. Team members were responsible to organize such presentations through any of their travel opportunities abroad.
\item Articles and/or blogposts about the project in team members' alma mater websites.
\item In-booth presentation and poster display in the seminars or career events at different universities. 
\item A thoroughly documented and user-friendly manual on how to build replicate and launch CAC and AAC sampling systems will be produced and published.
\end{itemize}