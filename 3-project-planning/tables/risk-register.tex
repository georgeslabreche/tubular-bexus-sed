


\begin{longtable}{|m{0.075\textwidth}| m{0.48\textwidth} |m{0.02\textwidth} |m{0.02\textwidth}|m{0.10\textwidth}| m{0.64\textwidth}|}

\hline
\textbf{ID} & \textbf{Risk (\& consequence if)} & \textbf{P} & \textbf{S} & \textbf{P * S} & \textbf{Action} \\ \hline
TC10 & Software fails to store data & B & 2 & \cellcolor[HTML]{34FF34}Very Low & Acceptable Risk: Extensive testing will be done. Using telemetry, all data gathered from sensors will be sent to ground station. \\ \hline
TC20 & Failure of several sensors & B & 2 & \cellcolor[HTML]{34FF34}Very Low & Acceptable Risk: Thermal test (Test Number 5) to approve the functionality of the experiment. \\ \hline
TC30 & Critical component is destroyed in testing & B & 1 & \cellcolor[HTML]{34FF34}Very Low & Acceptable Risk: Spare components can be ordered but for expensive ones, they will be ordered and tested early in the project in case we need to order more. \\ \hline
TC40 & Electrical connections dislodges or short circuits because of vibration or shock & B & 4 & \cellcolor[HTML]{FCFF2F}Low & Acceptable Risk. D-sub connections will be screwed in place. It will be ensured that there are no loose connections and zip ties will be used to help keep wires in place. Careful soldering and extensive testing will be applied. \\ \hline
TC50 & Experiment electronics fail due to long exposure to cold or warm temperatures & B & 3 & \cellcolor[HTML]{FCFF2F}Low & Acceptable Risk: Thermomechanical and thermoelectrical solutions will be simulated and tested in detail to help prevent this from happening. \\ \hline
TC60 & Software and electrical fail to control heaters causing temperature to drop or rise below or above operational range & B & 2 & \cellcolor[HTML]{34FF34}Very Low & Acceptable Risk: Tests will be performed prior to the flight to detect and minimize the risk of occurrence.The system will be monitored during flight and handled manually if necessary. \\ \hline
TC70 & Software fails to enter safe mode (may result in loss of data) & B & 1 & \cellcolor[HTML]{34FF34}Very Low & Acceptable Risk: Extensive testing will be done. \\ \hline
TC80 & On-board memory will be full (flight time longer than expected) & A & 2 & \cellcolor[HTML]{34FF34}Very Low & Acceptable Risk: The experiment shall go through testing and analysis to guarantee the onboard memory size is sufficient.\\ \hline
TC90 & Connection loss with ground station & A & 2 & \cellcolor[HTML]{34FF34}Very Low & Acceptable Risk: Experiment will be designed to operate autonomously. \\ \hline
TC100 & Software fails to control valves autonomously & B & 2 & \cellcolor[HTML]{34FF34}Very Low & Acceptable Risk: Extensive testing will be done. Telecommand will also be used to manually control the valves. \\ \hline
TC110 & Software fails to change modes autonomously & B & 2 & \cellcolor[HTML]{34FF34}Very Low & Acceptable Risk: Extensive testing will be done. Telecommand will also be used to manually change experiment modes. \\ \hline
TC120 & Complete software failure & B & 4 & \cellcolor[HTML]{FCFF2F}Low & Acceptable Risk: A long duration testing (bench test) will be performed to catch the failures early. \\ \hline
TC130 & Failure of fast recovery system & B & 2 & \cellcolor[HTML]{34FF34}Very Low & Acceptable Risk: Clear and simple instructions will be given to the recovery team. A test will take place before launch to ensure someone unfamiliar with the experiment can remove the CAC box. Test number: 12. \\ \hline
TC140 & The gas analyzer isn't correctly calibrated and returns inaccurate results & B & 3 & \cellcolor[HTML]{FCFF2F}Low & Acceptable Risk: Calibrate the gas analyzer before use.\\ \hline 
TC150 & Partnership with FMI does not materialize, resulting in loss of access to CAC coiled tube. & B & 2 & \cellcolor[HTML]{34FF34}Very Low & Acceptable Risk: Signed agreement has been obtained. AAC sample analysis results can be validated against available historical data from past FMI CAC flights. \\ \hline 
MS10 & Down link connection is lost prematurely & B & 2 & \cellcolor[HTML]{34FF34}Very Low & Acceptable Risk: Data will also be saved on SD card. \\ \hline
MS20 & Condensation on experiment PCBs which could causes short circuits & A & 3 & \cellcolor[HTML]{34FF34}Very Low & Acceptable Risk: The Brain will be sealed to prevent condensation. \\ \hline
MS30 & Temperature sensitive components that are essential to full the mission objective might be below their operating temperature. & C & 3 & \cellcolor[HTML]{FCFF2F}Low & Acceptable Risk: Safe mode to prevent the components to operate out of its operating temperature range. \\ \hline
MS40 & Experiment lands in water causing electronics failure & B & 1 & \cellcolor[HTML]{34FF34}Very Low & Acceptable Risk: Check if SD card needs waterproof shell or is waterproof in itself. Also, all the necessary data will be downloaded during the flight. \\ \hline
MS50 & Interference from other experiments and/or balloon & A & 2 & \cellcolor[HTML]{34FF34}Very Low & Acceptable Risk: no action. \\ \hline
MS60 & Balloon power failure & B & 2 & \cellcolor[HTML]{FCFF2F}Low & Acceptable Risk: Valves default state is closed so if all power is lost valves will automatically close preserving all samples collected up until that point. \\ \hline
MS70 & Sampling bags disconnect & B & 3 & \cellcolor[HTML]{FCFF2F}Low & Acceptable Risk: The affected bags could not collect samples. The connection between the spout of the bags and the T-union shall be double checked before flight. The system has passed vibration testing with no disconnects. \\ \hline
MS71 & Sampling bags puncture & B & 3 & \cellcolor[HTML]{FCFF2F}Low & Acceptable Risk: The affected bags could not collect samples. Inner styrofoam walls have been choosen and no sharp edges will be exposed to avoid puncture from external elements. \\ \hline
MS72 & Sampling bags' hold time is typically 48h & B & 2 & \cellcolor[HTML]{34FF34}Very Low & Acceptable risk: Validation studies have demonstrated acceptable stability for up to 48 hours.  \\ \hline
MS80 & Pump failure & B & 3 & \cellcolor[HTML]{FCFF2F}Low & Acceptable Risk: A pump was chosen based on a previous similar experiment. The pump has also been tested in a low pressure chamber down to 10hPa and has successfully turned on and filled a sampling bag. The CAC subsystem is not reliant on the pump therefore would still operate even in the event of pump failure. \\ \hline
MS90 & Intake pipe blocked by external element & C & 3 & \cellcolor[HTML]{FCFF2F}Low & Acceptable Risk: The bags would not be filled and thus the AAC system would fail. An air filter will be placed in both intake and outlet of the pipe to prevent this. \\ \hline
MS100 & Expansion/Contraction of insulation & B & 2 &\cellcolor[HTML]{34FF34}Very Low & Acceptable Risk: The insulation selected has flown successfully on similar flights in the past. Test shall be done to see how it reacts in a low pressure environment. \\ \hline
MS110 & Sampling bags are over-filled resulting in bursting and loss of collected samples. & B & 3 & \cellcolor[HTML]{FCFF2F}Low & Acceptable Risk: Test will be performed at target ambient pressure levels to identify how long the pump needs to fill the sampling bags. A static pressure sensor on board will monitor the in-bag pressure during sampling and no bag will ever be over pressured. In addition an airflow rate sensor will monitor the flow rate and a timer started when a bag valve is opened. The sampling will stop when either the maximum allowed pressure or maximum allowed time is reached. \\ \hline
SF10 & Safety risk due to pressurized vessels during recovery. & A & 1 & \cellcolor[HTML]{34FF34}Very Low & Acceptable Risk: The volume of air in the AAC decreases during descent because the pressure inside is lower than outside. The CAC is sealed at nearly sea level pressure, therefore there is only a small pressure difference.  \\ \hline
SF20 & Safety risk due to the use of chemicals such as magnesium perchlorate. & A & 4 & \cellcolor[HTML]{34FF34}Very Low & Acceptable Risk: The magnesium perchlorate will be kept in a sealed container or filter at all times. Magnesium perchlorate filters are made of stainless steel which has high durability, and have been used before without any sealing problems.  \\ \hline
VE10 & SD-card is destroyed at impact & B & 2 & \cellcolor[HTML]{34FF34}Very Low & Acceptable Risk: All data will be transmitted to the ground. Most of the data is the gas stored in the AAC and CAC. \\ \hline
VE20 & Gondola Fixing Interface & B & 4 & \cellcolor[HTML]{FCFF2F}Low & Acceptable Risk: The experiment box could detach from the gondola’s rails and the two boxes could detach one from the other. The experiment will be secured to the gondola and to each other with multiple fixings. These will also be tested. \\ \hline
VE30 & Structure damage due to bad landing & B & 3 & \cellcolor[HTML]{FCFF2F}Low & Acceptable Risk: Landing directly on a hard element could break the structure or the protective walls. Consistent design implemented to prevent it. \\ \hline
VE40 & Hard landing damages the CAC equipment & C & 3 & \cellcolor[HTML]{FCFF2F}Low & Acceptable Risk:  Structural analysis has been done and choosing a wall consisting of an aluminum sheet and Styrofoam to dampen the landing. \\ \hline
VE50 & Hard landing damages the AAC equipment & C & 3 & \cellcolor[HTML]{FCFF2F}Low & Acceptable Risk:  Structural analysis has been done and choosing a wall consisting of an aluminum sheet and Styrofoam to dampen the landing. \\ \hline
EN10 & Vibrations from pump affect samples & C & 1 & \cellcolor[HTML]{34FF34}Very Low & Acceptable Risk: Vibrations do not affect the sampled air. No action required. \\ \hline
EN20 & The air samples must be protected from direct sunlight and stored above 0$\degree{C}$ to prevent condensation & C & 3 & \cellcolor[HTML]{FCFF2F}Low & Acceptable Risk: Stratospheric air is generally dry and water vapor concentrations are higher closer to the surface. In addition magnesium perchlorate dryers will be used to minimizing the risk of condensation.    \\ \hline 
PE10 & Change in Project Manager after the CDR introduces a gap of knowledge in management responsibilities. & E & 1 & \cellcolor[HTML]{FCFF2F}Low & Acceptable Risk: A Deputy Project Manager is selected at an early stage and is progressively handed over project management tasks and responsibilities until complete handover after the CDR. The previous Project Manager remotely assists the new Project Manager until the end of the project. The Deputy Project Manager is also part of the Electrical Division so a new team member has been included to that division in order compensate for the Deputy Project Manager's reduced bandwidth to work on Electrical Division tasks once she is appointed Project Manager.\\ \hline 
PE20 & Team members from the same division are unavailable during the same period over the summer. & C & 2 & \cellcolor[HTML]{FCFF2F}Low & Acceptable Risk: Summer travel schedules have been coordinated among team members so that there is at least one member from each division available during the summer. \\ \hline
PE30 & No one from management is available to oversee the work for a reasonable period. & B & 2 & \cellcolor[HTML]{34FF34}Very Low & Acceptable Risk: Management summer travel schedules have been planned to fit around known deadlines. There will always be at least one member from management available via phone at all times. All team members are made aware of which members will be available at what times so work can be planned accordingly. \\ \hline
PE40 & Miscommunication between team members results in work being incomplete or inaccurate & B & 2 & \cellcolor[HTML]{34FF34}Very Low & Acceptable Risk: Whatsapp, Asana and Email are used in combination to ensure that all team members are up to date with the most current information. \\ \hline
% & & & & &\\ \hline

%\end{tabular}
\caption{Risk Register.}
\label{tab:risk-register}
\end{longtable}
\raggedbottom