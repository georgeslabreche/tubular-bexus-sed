
\subsection{Work Breakdown Structure}

The team is categorized into different groups of responsibilities with dedicated leaders who will report to and coordinate with the Project Manager. Leadership may be organized on a rotational basis should the need arise. The formation of these divisions constitute a work breakdown structure in which is illustrated in Figure \ref{fig:work-breakdown-structure}.


The interaction between the divisions will be refined over the course of project implementation to acknowledge the interdisciplinary nature of the experiment around a Payload / Platform scheme.

The Management is composed of a Project Manager and a Deputy Project Manager, both acting as Systems Engineer and managing overall implementation of the project. The Project Manager is responsible for establishing and overseeing product development cycle; coordinating between different teams, project stakeholders, and documentation efforts; outreach and public relations; Fundraising; monitoring and reporting; system integration; and quality assurance. Once all subsystems have been assembled, the Project Manager will be responsible for overseeing the integration processes leading to the final experiment setup and will put emphasis on leading quality assurance integration testing efforts. The Deputy Project Manager assists the Project Manager in all management duties in a manner that ensures replaceability when necessary.

The Scientific Division is responsible for defining experiment parameters; data analysis; interpreting and documenting measurements; researching previous CAC experiments for comparative analysis purposes; evaluating the reliability of the proposed AAC sampling system; conducting measurements of collected samples; documenting and publishing findings; defining experiment parameters; contacting researchers or institutions working on similar projects; exploring potential partnership with researchers and institutions; documenting and publishing findings.

The Mechanical Division is responsible for designing or redesigning cost-effective mechanical devices using analysis and computer-aided design; producing details of specifications and outline designs; overseeing the manufacturing process for the devices; identifying material and component suppliers; developing and testing prototypes of designed devices; analyzing test results and changing the design as needed; and integrating and assembling final design.

The Electrical Division is responsible for designing and implementing cost-effective circuitry using analysis and computer-aided design; producing details of specifications and outline designs; developing, testing, and evaluating theoretical designs; identifying material as well as component suppliers; reviewing and testing proposed designs; recommending modifications following prototype test results; and assembling designed circuitry.

The Software Division is responsible for gathering software requirements; formalizing software specifications; drafting architecture design; leading software implementation efforts; leading quality assurance and testing efforts; enforcing software testing best practices such as continuous integration testing and regression testing; reviewing requirements and specifications in order to foresee potential issues; providing input for functional requirements; advising on design; formalizing test cases; tracking defects and ensuring their resolution; facilitating code review sessions; and supporting software implementation efforts.

The Thermal Division is responsible for ensuring thermal regulation of the payload as per operational requirements of all experiment components; evaluating designs against thermal simulation and propose improvements; managing against mechanical design and electrical power limitations towards providing passive and active thermal control systems.

\begin{landscape}
\begin{figure}[p]
    \begin{align*}
        \includegraphics[width=24cm]{3-project-planning/img/work-breakdown-structure.png}
    \end{align*}
    \caption{Work Breakdown Structure.}\label{fig:work-breakdown-structure}
\end{figure}
\end{landscape}