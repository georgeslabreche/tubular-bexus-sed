\subsection{Experiment Objectives}

Beyond providing knowledge on greenhouse gas distributions, the sampling obtained from the experiment will serve as a reference to validate the robustness and reliability of the proposed sampling system through comparative analysis of results obtained with a reference sampling system.

The primary objective of the experiment consists of validating the proposed sampling system as a reliable mechanism that enables sampling of stratospheric greenhouse gases in remote areas. Achieving this objective consists of developing a cost-effective and re-usable stratospheric air sampling system (i.e. AAC). Samples collected by the proposed mechanism are to be compared against samples collected by a proven sampling system (i.e. CAC). The proven sampling system is to be part of the experimental payload as a reference that will validate the proof-of-concept air sampling system.

The secondary objective of the experiment will be to analyze the samples by both systems in a manner that will contribute to climate change research in the Arctic region. The trace gas profiles to be analyzed are that of carbon dioxide (CO$_{2}$), methane (CH$_{4}$), and carbon oxide (CO). The research activities will culminate in a research paper written in collaboration with FMI.