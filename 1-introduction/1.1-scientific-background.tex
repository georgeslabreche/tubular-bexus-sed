\subsection{Scientific Background}

The ongoing and increasingly rapid melting of the Arctic ice cap has served as a reference to the global climate change. Researchers have noted that \enquote{the Arctic is warming about twice as fast as the rest of the world} \cite{Perkins} and projecting an ice-free Arctic Ocean as a realistic scenario in future summers similar to the Pliocene Epoch when \enquote{global temperature was only 2–3°C warmer than today} \cite{Trace}. Suggestions that additional loss of Arctic sea ice can be avoided by reducing air pollutant and CO$_{2}$ growth still require confirmation through better climate effect measurements of CO$_{2}$ and non-CO$_{2}$ forcings \cite{Trace}. Such measurements bear high costs, particularly in air sampling for trace gas concentrations in the region between the upper troposphere and the lower stratosphere which have a significant effect on the Earth's climate. There is little information on distribution of trace gases at the stratosphere due to the inherent difficulty of measuring gases above aircraft altitudes.

Trace gases, are gases which makes up less than 1\% by volume of the Earth's atmosphere. They include all gasses except Nitrogen, and Oxygen. In terms of climate change, the main concern of the scientific community, focuses on CO$_2$ and CH$_4$ which make up less than 0.1\% of the 1\%, and are referred to as Greenhouse gases. Greenhouse gas concentrations are measured in parts per million (ppm), and parts per billion (ppb). They are the main offenders of the greenhouse effect, released by human activity as they trap heat into the atmosphere. Larger emissions of greenhouse gases lead to higher concentrations of those gases in the atmosphere thus contributing to climate change.
