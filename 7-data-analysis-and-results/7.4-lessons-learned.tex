\pagebreak
\subsection{Lessons Learned}
At this early stage of the experiment, and having already submitted an accepted experimental proposal, the TUBULAR team has learned important lessons regarding document creation as well as learning how to build an idea into a project. \par
The TUBULAR team expects that the BEXUS programme will be rewarding in terms of experience regarding balloon craft design and development, with real deadlines, published documents, and team work. This part of the document will be updated in later SEDs to reflect what the team members have learned.

\subsubsection{Science}
After an extended research in trace gases and climate change, as well as in atmospheric sampling methods, the science team has gained so far: 
\begin{itemize}
    \item General knowledge in climate change.
    \item General knowledge in the different sampling methods of the atmosphere; its characteristics and applications.
    \item Study scientific papers in detail.
    \item Outreach to scientific community.
    \item Translating scientific concepts to technical teams.
    \item Knowledge of how to design the scientific requirements in such a way that are in the permitted limits of the budget while the technical requirements are fulfilled.   
    \item How to sufficiently distribute the tasks within the science team and keep good communication with the other departments. 
    \item Experience, that writing down the tasks that need to be done, and keep tracking on them is better rather than having them as goals.   
    \item Knowledge in data analysis procedure and how to extract the desired results from raw data.
    \item That the distribution of information from each division is better to organized separately in each department's folder. This saves precious time, finding easier the correct piece of information when needed. 
\end{itemize}


\subsubsection{Electrical Division}
The electrical team has thus far enhanced its understanding of the electronics design as well as gained confidence in selecting appropriate components as per requirements. Some of the points team improved as their general understanding are listed below:  
\begin{itemize}

    \item Gained confidence in designing electronics circuitry.
    \item Familiarized with the selection of the electrical components. 
    \item By reading through large number of data sheets, team is now able to easily extract and understand technical details. 
    \item Learned and developed power calculation skills.
\end{itemize}


\subsubsection{Software Division}

\begin{itemize}
    \item Learned more about version control in the form of Git.
    \item Learned how to implement RTOS in an Arduino micro-controller.
    \item Learned how to translate experiment requirements to software design.
    \item Learned how to split functionality into several testable functions.
    \item Gained experience on software unit test.
    \item Learned how to design and create GUI using MATLAB.
    \item Learned how to use Git, a version control system for tracking changes in computer files and coordinating work on those files among multiple people.
\end{itemize}


\subsubsection{Mechanical Division}

\begin{itemize}
    \item Come up with real design solutions starting from conceptual problems.
    \item Make a proper use of both space and mass.
    \item Learn mechanical \textit{tricks} when designing. 
    
    % \item Choose components depending of availability and required characteristics.
\end{itemize}


\subsubsection{Management Division}

\begin{itemize}
    \item Coordination between multiple project stakeholders.
    \item Task definition, estimation, and management.
    \item Task integration.
    \item Conflict management and resolution.
    \item Communication flows.
    \item Funding research and outreach.
\end{itemize}

\subsubsection{Thermal Division}
\begin{itemize}
    \item Learned how to do Steady-State and transient thermal in ANSYS.
    \item Coordinate between other division to find a solution that works for everyone.
    \item 
\end{itemize}