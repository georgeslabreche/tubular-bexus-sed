\pagebreak
\subsection{Lessons Learned}
At the end of the build and test phase of the experiment and having already passed the integration progress review, the TUBULAR Team has learned important lessons regarding document creation as well as learning how to build an idea into a project. \par
The TUBULAR Team expects that the REXUS/BEXUS programme will be rewarding in terms of experience regarding balloon craft design and development, with real deadlines, published documents, and team work. This part of the document will be updated in later SEDs to reflect what the team members have learned.

\subsubsection{Management Division}

\begin{itemize}
    \item Coordination between multiple project stakeholders.
    \item Task definition, estimation, and management.
    \item Task integration.
    \item Conflict management and resolution.
    \item Communication flows.
    \item Funding research and outreach.
    \item Identifying team member strengths as well as weaknesses and assigning responsibilities accordingly without neglecting the opportunities to improve on weaknesses. 
    \item Do not assume cross-division communication will take place without organizing/planning it.
    \item Reviewing progress of assigned task should be continuous rather than waiting for their due dates.
    \item Agree on and clearly communicate to the team definition of \enquote{Done} when referring to tasks being completed.
    \item Agree on and clearly communicate to the team the definition of \enquote{Final Version} when referring to schematics, diagrams, and component lists.
    \item The lessons learned section of previous BEXUS SEDs is an invaluable resource that answers many BEXUS related recurring questions.
    \item If changes in management are required it is important that there is a sufficiently long change over period to allow a transfer of knowledge.
    \item Tasks that are not completed on time or were simply not worked on during the assigned time will impact projected deadlines and these situations must be planned for and mitigated against. An early red flag for this is if the reported team working hours tend to be lower than expected at which point one can expect to have to make up those hours up before a deadline. These concerns must continuously be communicated to the team.
    \item The REXUS/BEXUS programme is a significant investment in time and resources from all programme partners and as such the unique opportunity is not limited to participating students but to component manufacturers and suppliers as well. With this in mind, the team should not shy away from aggressively seeking funds or sponsorships from component manufacturers and suppliers as they stand to benefit from such a partnership to show case the robustness of their products.
    \item Testing will always take longer than expected and so time must be planned to account for this.
    \item When working with many remote team members extra time must be allowed for tasks to be completed as the communication is slower. Internal earlier deadlines help a lot.
    \item During manufacture and test having many smaller deadlines has proven useful in ensuring things stick to the time plan.
\end{itemize}

\subsubsection{Scientific Division}
After an extended research in trace gases and climate change, as well as in atmospheric sampling methods, the science team has gained so far: 
\begin{itemize}
    \item General knowledge in climate change.
    \item General knowledge in the different sampling methods of the atmosphere; its characteristics and applications.
    \item Study scientific papers in detail.
    \item Outreach to scientific community.
    \item Translating scientific concepts to technical teams.
    \item Knowledge of how to design the scientific requirements in such a way that are in the permitted limits of the budget while the technical requirements are fulfilled.   
    \item How to sufficiently distribute the tasks within the science team and keep good communication with the other departments. 
    \item Experience, that writing down the tasks that need to be done, and keep tracking on them is better rather than having them as goals.   
    \item Knowledge in data analysis procedure and how to extract the desired results from raw data.
    \item Experience in producing a presentation only with the key points of a project and presenting it in front of other people.
    \item Work as a group from different locations.
    \item The importance of testing, and how to sufficiently deal with problems that come up unexpectedly.
     
\end{itemize}


\subsubsection{Electrical Division}
The electrical team has enhanced its understanding of the electronics design as well as gained confidence in selecting appropriate components as per requirements. Some of the points team improved as their general understanding are listed below:  
\begin{itemize}

    \item Gained confidence in designing electronics circuitry.
    \item Familiarized with the selection of the electrical components. 
    \item By reading through large number of data sheets, team is now able to easily extract and understand technical details. 
    \item Learned and developed power calculation skills.
    \item Got experience of using the Eagle software and how to find and make the libraries, footprints, and schematics for the required components.
    \item How to test the components in the vacuum chamber.
    \item Learned about the different connectors, wires and how to place the components on the PCB so the actual design can fit into the experiment box.
    \item Discovered the cascading consequences of changing one component.
    \item Finding how having big sheets with a lot of information can be preferable to several sheets with less specification.
    \item While designing PCB's with Eagle it's a good idea to draw the traces manually rather than using autotracer. Since it allows you double check your schematics while pulling the traces.
    \item When using netnaming to design schematics for later PCB designs in Eagle it's very important to triple check the net names since they sometimes change in unexpected ways.
    \item Got practical experience of soldering the different types of sensors, wires and connectors. 
    \item Learned how to sold SMD miniature pressure sensors onto the PCB.
    \item Familiarized with using work shop tools and machinery.
    
\end{itemize}


\subsubsection{Software Division}

\begin{itemize}
    \item Learned more about version control in the form of Git.
    \item Learned how to implement an RTOS on Arduino.
    \item Learned how to translate experiment requirements to a software design.
    \item Learned how to split functionality into several testable functions.
    \item Gained experience on software unit testing.
    \item Learned how to design and create GUI using MATLAB GUIDE.
    \item Learned how to use Git, a version control system for tracking changes in computer files and coordinating work on those files among multiple people.
    \item Learned how to implement TCP/IP and UDP on ethernet connection.
    \item Learned how to make telecommand and telemetry.
    \item Learned how the I2C and SPI protocols work and operate.
    \item Learned how to efficiently debug software. 
    %\item Learned how 
\end{itemize}


\subsubsection{Mechanical Division}

\begin{itemize}
    \item Come up with real design solutions starting from conceptual problems.
    \item Make a proper use of both space and mass.
    \item Learn mechanical \textit{tricks} when designing.
    \item Adapt the design to components availability and characteristics.
    \item Select and contact with vendors.
    \item Implement a real pneumatic system. 
    \item Compute structural analysis.
    \item Team collaboration with other departments, i.e. Electrical, Science, and Thermal.
    \item Design is trickier when it comes to implementation.
    \item Always document specific department knowledge. If who designed a certain part of the experiment is not available for the manufacture phase, whoever works on it should be able to figure out most of the solutions by themselves.
    \item Manufacturing and integration of the different subsystems of the experiment takes longer than expected during design phase.
    \item The design is never frozen until everything is built and working properly.
\end{itemize}

\subsubsection{Thermal Division}
\begin{itemize}
    \item Learned how to do Steady-State and Transient thermal analysis in ANSYS.
    \item Coordinate between other division to find a solution that works for everyone.
    \item Do a thermal plan and structure up what needs to be done for a long period of time.
    \item How to improve and be more efficient when adjusting to sudden changes in design.
    \item How to balance details in simulations.
    \item How to do thermal test, analyze the result and make improvement of the results.
    \item How to work with Styrofoam.
    \item How temperatures inside component operating ranges can impact component performances.
\end{itemize}