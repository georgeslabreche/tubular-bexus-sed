\newpage
\section{Appendix I - Experiment Thermal Analysis} \label{sec:appI}
\subsection{Thermal equations}

\subsubsection{Variables and Tables}
\begin{table}[H]
    \centering
    \begin{tabular}{|c|c|c|c|}
        \hline
        \textbf{Variable} & \textbf{Description} & \textbf{Unit} & \textbf{Value} \\ \hline
        $\alpha_{Al}$ & Absorption of aluminum & - & 0.3 \\ \hline
        S & Solar constant & $\frac{W}{m^2}$ & 1362 \\ \hline
        $A_{Sun}$ & Area affected by the sun & $m^2$ & 0.28 \\ \hline
        Albedo & Albedo coefficient & - & 0.15 \\ \hline
        $A_{Albedo}$ & Area affected by the albedo & $m^2$ & 0.65 \\ \hline
        $\varepsilon_{Earth}$ & Emissivity of Earth & - & 0.95 \\ \hline
        $A_{IR}$ & Area affected by the IR flux & $m^2$ & 0.65 \\ \hline
        $IR_{25km}$ & Earth IR flux at 25 km & $\frac{W}{m^2}$ & 220 \\ \hline
        P & Dissipated power from electronics & W & varies \\ \hline
        h & Convection heat transfer constant & $\frac{W}{m^2 \cdot K}$ & 18 \\ \hline
        K & Scaling factor for convection & - & varies \\ \hline
        $A_{Convection}$ & Area affected by the convection & $m^2$ & 1.3 \\ \hline
        $\sigma$ & Stefan-Boltzmann constant & $\frac{W}{m^2 \cdot K^4}$ & $5.67051 \cdot 10^{-8}$ \\ \hline
        $A_{Radiation}$ & Radiating area & $m^2$ & 1.3\\ \hline
        $\varepsilon_{Al}$ & Emissivity of aluminum & - & 0.09 \\ \hline
        $T_{Out}$ & Temperature wall outside & $K$ & varies \\ \hline
        $T_{Inside}$ & average uniform temperature inside & $K$ & varies \\ \hline
        $T_{Ambient}$ & Ambient temperature outside & $K$ & varies \\ \hline
        $T_{Ground}$ & Temperature of the ground & $K$ & 273 \\ \hline
        $k_{Al}$ & Thermal conductivity of aluminum & $\frac{W}{m\cdot K}$ & 205 \\ \hline
        $k_{PS}$ & Thermal conductivity of polystyrene foam & $\frac{W}{m\cdot K}$ & 0.03 \\ \hline
        $L_{Al}$ & Thickness of aluminum sheeting & $m$ & 0.0005 \\ \hline
        $L_{PS}$ & Thickness of polystyrene foam & $m$ & varies \\ \hline
        $P_{Ground}$ & Pressure at ground & $Pa$ & $101.33 \cdot 10^3$ \\ \hline
        $P_{25km}$ & Pressure at $25 km$ & $Pa$ & $2.8 \cdot 10^3$ \\ \hline
    \end{tabular}
    \caption{Variables Used in Thermal Calculation}
    \label{tab:thermal-variables}
\end{table}

\begin{table}[H]
\centering

\begin{tabular}{ll}
\hline
\multicolumn{1}{|l|}{\textbf{Wall part}}              & \multicolumn{1}{l|}{\textbf{Thickness (m)}} \\ \hline
\multicolumn{1}{|l|}{Aluminum sheet}         & \multicolumn{1}{l|}{0.0005}        \\ \hline
                                             &                                    \\ \cline{1-1}
\multicolumn{1}{|l|}{AAC (Styrofoam)}        &                                    \\ \hline
\multicolumn{1}{|l|}{Vertical}               & \multicolumn{1}{l|}{0.02}          \\ \hline
\multicolumn{1}{|l|}{Horizontal}             & \multicolumn{1}{l|}{0.02}          \\ \hline
\multicolumn{1}{|l|}{Top/Bottom}             & \multicolumn{1}{l|}{0.03}          \\ \hline
                                             &                                    \\ \cline{1-1}
\multicolumn{1}{|l|}{CAC (Styrofoam)}        &                                    \\ \hline
\multicolumn{1}{|l|}{Horizontal towards AAC} & \multicolumn{1}{l|}{0.02}          \\ \hline
\multicolumn{1}{|l|}{All other walls}        & \multicolumn{1}{l|}{0.05}          \\ \hline
\end{tabular}
\caption{The Different Wall Thicknesses Used for AAC and CAC}
\label{tab:Wall-thickness-AAC-CAC}
\end{table}

%\begin{table}[H]
%\centering
%\caption{Dissipated power at the different stages.}
%\label{tab:dissipated-power-thermal}
%\begin{tabular}{|l|l|l|}
%\hline
%\multirow{2}{*}{Critical stage} & \multicolumn{2}{l|}{Dissipated power (W)} \\ \cline{2-3} 
%                                & Worst case       & Average case       \\ \hline
%Launch pad                      & 7.589            & 5.083              \\ \hline
%Early ascent                    & 11.499           & 8.993              \\ \hline
%Sampling ascent                 & 13.499           & 13.397             \\ \hline
%Float                           & 11.499           & 8.993              \\ \hline
%Sampling descent                & 13.499           & 13.397             \\ \hline
%Shutdown descent                & 2.167            & 2167               \\ \hline
%Landed                          & 0                & 0                  \\ \hline
%\end{tabular}
%\end{table}
%The difference in dissipated power is depending if heaters need to be on or not.


\subsection{Thermal calculations in MATLAB}
For the MATLAB calculations a few assumptions were made, they are as follows.
\begin{itemize}
    \item Taking the average of MATLAB calculations for calculations with or without sun.
    \item Calculate the average temperature on the outside wall of the experiment.
    \item Assuming the inner temperature at the bags section is uniform.
    \item The pipes letting cold air in have not been taken into account in MATLAB.
    \item Assume no interference between the two experiment boxes.
    \item All conduction is uniform from the inside.
    \item Assume steady flow through the walls from conduction.
    \item Assume radiation and convection from/on 6 walls not 5.
\end{itemize}

\subsubsection{Solar flux and Albedo}
The albedo is the reflected solar flux from earth so it can be put into the same equation as the solar flux. It is assumed that the sun hit two sides of the experiment at a $45\degree$ angle at all time over 10 km. In the mid of October at the time for launch the sun will hit the experiment with a maximum inclination of $15\degree$ from the horizon.
\begin{equation*}
    Q_{Sun+Albedo} = \alpha_{Al}\cdot S \cdot cos(15) \cdot (A_{Sun} \cdot cos(45) + Albedo \cdot A_{Albedo})
\end{equation*}

\subsubsection{Conduction}
For calculating the outer walls temperature, the assumption of steady flow through walls is used.
\begin{equation*}
    Q_{Conduction} = [\text{Steady flow through wall}] = \text{Dissipated power} = P
\end{equation*}

\subsubsection{Earth IR flux}
The earth IR flux is the flux that comes from earth as a black body radiating. It is calculated from finding the IR flux at the ground then scale it to the altitude the experiment will fly at. The following equations were found from \cite{BalloonAscent}.
\begin{gather*}
    IR_{Ground} = \varepsilon_{earth} \cdot \sigma \cdot T_{ground}^4 \\
    \tau_{atmIR} = 1.716 - 0.5\cdot \Bigg[e^{-0.65\frac{P_{25km}}{P_{ground}}} + e^{-0.95\frac{P_{25km}}{P_{ground}}}\Bigg] \\
    IR_{25km} = \tau_{atmIR} \cdot IR_{Ground}
\end{gather*}
After the IR has been calculated for the floating altitude it is put into the following equation. 
\begin{equation*}
    Q_{IR} = \varepsilon_{earth} \cdot A_{IR} \cdot IR_{25km}
\end{equation*}

\subsubsection{Radiation}
It is assumed that the experiment will experience radiation from all 6 sides. In reality it will experience radiation from 5 sides because the CAC box will be in contact with one of the AAC box's sides. It was decided to leave it at 6 for the calculations in order to compensate for having no holes to let cold air in to the pump.
\begin{equation*}
    Q_{Radiation} = \sigma \cdot \varepsilon_{Al} \cdot A_{Radiation} \cdot (T_{Out}^4 - T_{Ambient}^4 )
\end{equation*}

\subsubsection{Convection}
At an altitude of 25 km there is far lower air density than at sea level. Therefore, it gives that a scaling factor $K$ has to be taken into account when calculating the convection and K can be seen in Table \ref{tab:heat-loss} for different altitudes.
\begin{equation*}
    Q_{Convection} = h \cdot K \cdot A_{Convection} \cdot (T_{Out} - T_{Ambient})
\end{equation*}

The equation for approximating the heat transfer coefficient for air is outlined as:
\begin{equation*}
h = 10.45 - v + 10\cdot\sqrt{v}
\end{equation*}
Where $v$ is the velocity of the fluid medium.

As the balloon is expected to rise at approximately $5 m/s$ for the duration of the Ascent Phase, the starting value for the convective heat transfer coefficient $h$ is expected to be $27.811$, assuming negligible wind currents perpendicular to the direction of ascent. %I will probably move this paragraph to our Thermal Design section, since this is important information.

The equations used to obtain the value of $K$ are listed below:

\begin{equation*}
F(T_{sea}, T_{alt}) = \big(\frac{k_{alt}}{k_{sea}}\big)^{1-n}\times \Big[\Big(\frac{\beta_{alt}}{\beta_{sea}}\Big)\times \Big(\frac{\mu_{sea}}{\mu_{alt}}\Big)\times \Big(\frac{c_{p-alt}}{c_{p-sea}}\Big)\times \Big(\frac{\rho(T_{alt})}{\rho(T_{sea})}\Big)^{2}\Big]^{n}
\end{equation*}

Where:
\begin{itemize}
    \item $n$ is an exponent value dependent on the turbulence of the fluid medium ($\frac{1}{4}$ for laminar flow and $\frac{1}{3}$ for turbulent flow)
    \item $k$ is the thermal conductivity of the air
    \item $\beta$ is the thermal expansion coefficient for air
    \item $\mu$ is the dynamic viscosity of the air
    \item $c_{p}$ is the specific heat capacity of the air at constant pressure
    \item $\rho(T)$ is the density of the air as a function of only temperature difference (i.e. for constant pressure)
    \item "sea" denotes the current variable is represented by its value found at sea level
    \item "alt" denotes the current variable is represented by its value found at a specified altitude
\end{itemize}


The values for $F$ from this equation were then applied to its respective position in the following equation to determine the ratio between the convective heat transfer coefficient $h$ at sea level (assumed to have negligible differences for Esrange ground level) and the same coefficient at a specified altitude:

\begin{equation*}
K = \Big(\frac{\rho(P_{alt})}{\rho(P_{sea})}\Big)^{2n}\times \Big(\frac{\Delta T_{air}}{\Delta T_{sea}}\Big)^{n}\times F(T_{sea}, T_{alt})
\end{equation*}

Where:
\begin{itemize}
    \item $\rho(T)$ is the density of the air as a function of only temperature difference (i.e. for constant pressure)
    \item $\delta T$ is the difference between the temperature of the ambient air and the surface in question
\end{itemize} \\


%Table for convective and radiative heat loss will go here.

Table \ref{tab:heat-loss} combines the previously listed convection and radiation formulae integrated into the MATLAB scripts to determine the convective and radiative heat loss in the worst case for (highest) power dissipation during each stage of the experiment. Additional information on the thermodynamics of the atmosphere was obtained from \textit{Engineering Toolbox} \cite{EngTool} \\





\begin{longtable}{|m{2.5cm}|m{1.6cm}|m{1cm}|m{1.2cm}|m{1.2cm}|m{1cm}|m{1.5cm}|m{1.5cm}|}
\hline

Altitude & Case & $T_{amb}$ & $K$ & $h_{alt}$ & $T_{out}$ & $Q_{conv}$ & $Q_{rad}$ \\ \hline
\multirow{3}{*}{\begin{tabular}[c]{@{}l@{}}Hangar\\ (Preparations)\end{tabular}} & Cold & 283 & 1 & 10.45 & 19.7 & 130.076 & 6.062 \\
 & Expected & 288 & 1 & 10.45 & 24.7 & 129.770 & 6.368 \\
 & Warm & 293 & 1 & 10.45 & 29.7 & 129.455 & 6.683 \\ \hline
\multirow{3}{*}{\begin{tabular}[c]{@{}l@{}}Ground\\ (Stationary)\end{tabular}} & Cold & 263 & 1 & 18 & -0.8 & 211.174 & 4.589 \\
 & Expected & 273 & 1 & 18 & 9.1 & 210.651 & 5.110 \\
 & Warm & 283 & 1 & 18 & 19.1 & 210.092 & 5.667 \\ \hline
\multirow{3}{*}{\begin{tabular}[c]{@{}l@{}}Ground\\ (Launched)\end{tabular}} & Cold & 263 & 1 & 27.811 & -3.9 & 216.694 & 2.995 \\
 & Expected & 273 & 1 & 27.811 & 6.1 & 216.348 & 3.340 \\
 & Warm & 283 & 1 & 27.811 & 16.1 &  215.978 & 3.710 \\ \hline
\multirow{3}{*}{5 km} & Cold & 228 & 0.7868 & 21.882 & -37.3 & 217.161 & 2.526 \\
 & Expected & 263 & 0.8468 & 23.550 & -2.8 &  216.135 & 3.550 \\
 & Warm & 273 & 0.8507 & 23.659 & 7.1 & 215.748 & 3.936 \\ \hline
\multirow{3}{*}{10 km} & Cold & 193 & 0.4882 & 13.577 & -67.5 & 217.088 & 2.584 \\
 & Expected & 223 & 0.5286 & 14.701 & -38.5 & 216.076 & 3.592 \\
 & Warm & 238 & 0.5421 & 15.076 & -23.9 & 215.449 & 4.218 \\ \hline
\multirow{3}{*}{15 km} & Cold & 193 & 0.3300 & 9.178 & -61.5 & 217.622 & 4.013 \\
 & Expected & 233 & 0.3680 & 10.234 & -23.6 & 215.586 & 6.043 \\
 & Warm & 253 & 0.3825 & 10.638 & -4.3 & 214.317 & 7.307 \\ \hline
\multirow{3}{*}{20 km} & Cold & 213 & 0.2401 & 6.677 & -35.0 & 214.021 & 7.526 \\
 & Expected & 243 & 0.2563 & 7.128 & -6.9 & 211.522 & 10.015 \\
 & Warm & 268 & 0.2687 & 7.473 & 16.8 & 209.114 & 12.413 \\ \hline
\multirow{3}{*}{25 km} & Cold & 223 & 0.1683 & 4.681 & -15.2 & 208.651 & 12.705 \\
 & Expected & 253 & 0.1792 & 4.984 & 12.1 & 204.939 & 16.404 \\
 & Warm & 273 & 0.1847 & 5.137 & 30.7 & 202.029 & 19.300 \\ \hline
\multirow{3}{*}{\begin{tabular}[c]{@{}l@{}}Floating\\ Phase\end{tabular}} & Cold & 223 & 0.1683 & 3.029 & 1.0 & 198.158 & 20.713 \\
 & Expected & 253 & 0.1792 & 3.226 & 26.6 & 192.931 & 25.942 \\
 & Warm & 273 & 0.1847 & 3.325 & 44.3 & 188.879 & 29.985 \\ \hline
 \multirow{3}{*}{25 km} & Cold & 223 & 0.1683 & 5.106 & -18.2 & 208.025 & 11.389 \\
 & Expected & 253 & 0.1792 & 5.436 & 9.4 & 204.618 & 14.782 \\
 & Warm & 273 & 0.1847 & 5.603 & 28.1 & 201.942 & 17.445 \\ \hline
 \multirow{3}{*}{20 km} & Cold & 213 & 0.2401 & 7.284 & -37.2 & 212.817 & 6.756 \\
 & Expected & 243 & 0.2563 & 7.775 & -8.9 & 210.536 & 9.027 \\
 & Warm & 268 & 0.2687 & 8.151 & 14.9 & 208.335 & 11.221 \\ \hline
 \multirow{3}{*}{15 km} & Cold & 193 & 0.3300 & 10.011 & -63.0 & 218.003 & 3.643 \\
 & Expected & 233 & 0.3680 & 11.164 & -24.9 & 216.134 & 5.506 \\
 & Warm & 253 & 0.3825 & 11.604 & -5.5 & 214.967 & 6.669 \\ \hline
 \multirow{3}{*}{10 km} & Cold & 193 & 0.4882 & 14.810 & -68.4 & 219.301 & 2.376 \\
 & Expected & 223 & 0.5286 & 16.036 & -39.4 & 218.365 & 3.310 \\
 & Warm & 238 & 0.5421 & 16.445 & -24.7 & 217.784 & 3.889 \\ \hline
 \multirow{3}{*}{5 km} & Cold & 228 & 0.7868 & 23.868 & -38.2 & 208.151 & 2.206 \\
 & Expected & 263 & 0.8468 & 25.689 & -3.7 & 207.250 & 3.105 \\
 & Warm & 273 & 0.8507 & 25.807 & 6.3 & 206.909 & 3.446 \\ \hline
 \multirow{3}{*}{\begin{tabular}[c]{@{}l@{}}Ground\\ (Landed)\end{tabular}} & Cold & 263 & 1 & 30.336 & -4.7 & 207.738 & 2.622 \\
 & Expected & 273 & 1 & 30.336 & 5.3 & 207.435 & 2.924 \\
 & Warm & 283 & 1 & 30.336 & 15.3 & 207.109 & 3.249 \\ \hline
 \multirow{3}{*}{\begin{tabular}[c]{@{}l@{}}Ground\\ (Stationary)\end{tabular}} & Cold & 263 & 1 & 18 & -1.2 & 203.752 & 4.420 \\
 & Expected & 273 & 1 & 18 & 8.8 & 203.248 & 4.922 \\
 & Warm & 283 & 1 & 18 & 18.7 & 202.710 & 5.458 \\ \hline
\caption{Table of Predicted Heat Loss}
\label{tab:heat-loss}
\end{longtable}

\raggedbottom


\subsubsection{Thermal equation}
If there is no sun on the experiment.
\begin{gather*}
    Q_{IR} + Q_{Conduction} = Q_{Radiation} + Q_{Convection} \\
    \updownarrow \\
    \varepsilon_{earth} \cdot A_{IR} \cdot IR_{25km} + P \\ = \sigma \cdot \varepsilon_{Al} \cdot A_{Radiation} \cdot (T_{Out}^4 - T_{Ambient}^4 ) + h \cdot K \cdot A_{Convection} \cdot (T_{Out} - T_{Ambient})
\end{gather*}
If there is sun on the experiment it is the same but adding $Q_{Sun+Albedo}$.
\begin{gather*}
    Q_{IR} + Q_{Conduction} + Q_{Sun+Albedo} = Q_{Radiation} + Q_{Convection} \\
    \updownarrow \\
    \varepsilon_{earth} \cdot A_{IR} \cdot IR_{25km} + P + \alpha_{Al}\cdot S \cdot cos(15) \cdot (A_{Sun} \cdot cos(45) + Albedo \cdot A_{Albedo}) \\ = \sigma \cdot \varepsilon_{Al} \cdot A_{Radiation} \cdot (T_{Out}^4 - T_{Ambient}^4 ) + h \cdot K \cdot A_{Convection} \cdot (T_{Out} - T_{Ambient})
\end{gather*}
From those equations $T_{Out}$ can be calculated and it is the average temperature on the aluminum sheets facing the outside air.
After $T_{Out}$ have been found the inner temperature can be calculated by using heat transfer through the wall.
\begin{gather*}
    P = \frac{T_{Inside} - T_{Outside}}{A \cdot (\frac{L_{Al}}{k_{Al}} + \frac{L_{PS}}{k_{PS}})} \\
     \updownarrow \\
    T_{Inside} = P \cdot A \cdot (\frac{L_{Al}}{k_{Al}} + \frac{L_{PS}}{k_{PS}}) + T_{Outside}
\end{gather*}
$T_{Inside}$ is then assumed to be the uniform air temperature in the experiment.



\subsubsection{Trial run with BEXUS 25 air temperature data for altitudes}
The air temperature data varying over altitude from old BEXUS flight could be found on the REXUS/BEXUS website. To do a simulated test flight for the calculations done in MATLAB, to see how it would be for a real flight it was calculated and plotted in with data from BEXUS 25 flight. 
Because of there being approx 42000 data points it had to be scaled down and only every $25^{th}$ data point was used to save time and there was not much detail loss by taking every $25^{th}$. In Figure \ref{fig:thermal-testflight-AAC} the TUBULAR test flight is the uniform temperature on the inside with a insulation consisting as specified in Table \ref{tab:Wall-thickness-AAC-CAC}.

\begin{figure}[H]
    \begin{align*}
        \includegraphics[width=1\linewidth]{appendix/img/Thermal/AAC-test-flight.jpg}
    \end{align*}
    \caption{Simulated Test Flight of TUBULAR AAC Box with Data From BEXUS 25}
    \label{fig:thermal-testflight-AAC}
\end{figure}
When the data was found it was checked in ANSYS to determine and add heaters to control the most critical parts of the model.

\subsubsection{Trial flight for the CAC} \label{sssec:CAC-trial-flight}
The CAC box does not require as much thermal design as the AAC box. The only part to consider is the valve. witch has a lower limit of the operating temperature of $-20\degree C$. It will not be a problem because the valve will open a little before launch and have a current throughout the whole flight heating it self up. If the thermal analysis is proven wrong by a test, showing that it is not sufficient to use only self heating, a heater can be applied at a later date. The passive thermal design for the CAC box will consist of aluminum sheets and Styrofoam as specified in \ref{tab:Wall-thickness-AAC-CAC}.
\begin{figure}[H]
    \begin{align*}
        \includegraphics[width=1\linewidth]{appendix/img/Thermal/CAC-test-flight.jpg}
    \end{align*}
    \caption{Simulated Test Flight of TUBULAR CAC Box with Data From BEXUS 25}
    \label{fig:thermal-testflight-CAC}
\end{figure}

\subsubsection{MATLAB Conclusion}
By running the MATLAB script, the hottest and coldest case for $0.02 m$ on the wall and $0.03 m$ on the top and bottom of the Styrofoam could be found for ascent and descent sampling. The thermal conductivity of Styrofoam is $k=0.03$. In Table \ref{tab:temperature-sampling-ascent-descent} it is shown the hottest and coldest case of temperature on the inside when samples should be taken. The hottest and coldest cases are taken from Figure \ref{fig:thermal-testflight-AAC}.


\begin{table}[H]
\centering
\begin{tabular}{l|l|l|l|l|}
\cline{2-5}
\multirow{2}{*}{}               & \multicolumn{2}{l|}{\textbf{Ascent}} & \multicolumn{2}{l|}{\textbf{Descent}} \\ \cline{2-5} 
                                & Coldest      & Hottest      & Coldest       & Hottest      \\ \hline
\multicolumn{1}{|l|}{AAC}       & -11.39       & 16.41        & -30.28        & -4.393       \\ \hline
\multicolumn{1}{|l|}{Outer air} & -38.22       & -15.9        & -44.41        & -38.18       \\ \hline
\end{tabular}
\caption{The Sampling Temperature Ranges for Ascent and Descent for the AAC Box}
\label{tab:temperature-sampling-ascent-descent}
\end{table}


\subsection{Thermal Simulations in ANSYS}
In ANSYS, FEA simulations were done using both Steady-State Thermal and Transient Thermal analysis.
Because of the limitations in ANSYS student license a simplified model has been used, which can be seen in Figure \ref{fig:Ansys-CAD-model}. It is in a lower corner of the experiment showing the Brain and has three walls to the sampling bags and the air is uniform on the inside. The uniform inside air can be taken from the data from the test flight in Figure (\ref{fig:thermal-testflight-AAC}). These simulations were done to see what temperature the pump and manifolds will be as they are the most critical components.

\begin{figure}[H]
    \begin{align*}
        \includegraphics[width=0.5\linewidth]{appendix/img/Thermal/CAD-ansys.JPG}
    \end{align*}
    \caption{The CAD Model Used for ANSYS Simulations}
    \label{fig:Ansys-CAD-model}
\end{figure}

The CAD model is as seen in the figure \ref{fig:Ansys-CAD-model}. The side exterior walls are $0.02 m$, the interior walls of the Brain to the bags are $0.03 m$ and the top and bottom wall consist of $0.03 m$ Styrofoam as well. The outer parts of the pipes are set to stainless steel with a constant temperature (the same as the ambient outside). The tubes closest to the pump and the one going from pump to the manifold were set to air to simulate and be able to vary depending on the temperature outside and the pump heating up from the heater.

A transient thermal analysis was also performed by simulating a test flight with data from BEXUS 25 using results from MATLAB. It was performed so the thickness of the wall could be verified to see if it was good enough and whether adding heaters was required. By being flexible with adding heaters, moving them around, changing their strength and their on time, it is possible to enable the pump and the manifold to operate in their required temperature ranges.

\begin{figure}[H]
    \begin{align*}
        \includegraphics[width=0.5\linewidth]{appendix/img/Thermal/Air-inside-AAC-sampling-ascent.JPG}
    \end{align*}
    \caption{Cross Section of the Air in the Brain at the Time to Sample During Ascent}
    \label{fig:Ansys-CAD-model}
\end{figure}

\begin{figure}[H]
    \centering
    \subfloat{\includegraphics[width=0.5\linewidth]{appendix/img/Thermal/Pump-sampling-ascent.JPG}}
    \hifll
    \subfloat{\includegraphics[height=0.2\linewidth]{appendix/img/Thermal/Valve-manifold-sampling-ascent.JPG}}
    \caption{Pump and Manifold at the Time to Sample During Ascent}
    \label{fig:Pump-Valve-ascent-sample}
\end{figure}

\begin{figure}[H]
    \centering
    \subfloat{\includegraphics[width=0.5\linewidth]{appendix/img/Thermal/Pump-sampling-descent.JPG}}
    \hifll
    \subfloat{\includegraphics[height=0.20\linewidth]{appendix/img/Thermal/Valve-manifold-sampling-descent.JPG}}
    \caption{Pump and Manifold at the Time to Sample During Descent}
    \label{fig:Pump-Valve-ascent-sample-descent}
\end{figure}

\begin{figure}[H]
    \centering
    \subfloat{\includegraphics[width=0.43\linewidth]{appendix/img/Thermal/Critical-lowest-pump.JPG}}
    \hifll
    \subfloat{\includegraphics[width=0.45\linewidth]{appendix/img/Thermal/Critical-lowest-valve.JPG}}
    \caption{Pump and Manifold at the Coldest Part of Ascent}
    \label{fig:Pump-Valve-ascent-critical-lowest}
\end{figure}

\subsection{Result}
The main objective from performing first the MATLAB calculations and then the ANSYS simulations has been to iterate and find the wall thickness of Styrofoam between the Brain and the inside of the AAC box. The next objective was to iterate the design with adding heaters to find the required amount and find approximately how long they need to run. By running a transient thermal analysis for the test flight there was the possibility to simulate heaters that will be on and off to determine how strong they need to be.

The results from the ANSYS simulations assumed a worst case scenario. It was to be expected that the results were not fully accurate, and instead were slightly warmer in reality. Figures \ref{fig:Pump-Valve-ascent-sample} and \ref{fig:Pump-Valve-ascent-sample-descent} show that the temperature of the pump is above $5\degree C$ and the manifold is above $-20\degree C$. It is only during a portion of the Ascent Phase, just prior to the start of sampling that the heater should be on in order for the pump to be above $5\degree C$, and it shall only be needed to be on during this Phase.

The insulation for the AAC will be as specified in Table \ref{tab:Wall-thickness-AAC-CAC}. For the three inner walls between the Brain and the bags there will be a $0.03 m$ wall of Styrofoam. By using two $5\,W$ heaters for the pump, one on top and one on bottom side and a 5 W heater for the manifold. The thermal simulations predict that they will be within the operating limits with a satisfactory margin. For the heater controller, it will be set that if the pump is below $15\degree\,C$ it will turn on. As for the manifold, the heater will be set to turn on if the manifold is below $-5 \degree\,C$.