\subsection{Post Flight Activities}

\subsubsection{AirCore Recovery}
To minimize the length of time in which mixing of the gas occurs in the collected CAC sample, it is necessary that the sample is analyzed within five to six hours after the experiment landing. Measurements will be made in partnership with the Finnish Meteorological Institute (FMI) thus it is necessary that the experiment can be transported to the measurement equipment provided by FMI. To analyze the samples at the landing site would be unfeasible due to the size of the gas analyzer, hence the analysis is to be done at the Esrange Space Centre.

%The FMI team would begin their travel to Esrange as soon as the balloon landing site has been located.  Once the experiment has been recovered to Esrange, the samples would be analyzed by the FMI equipment and the preliminary results would be found.  The disadvantages of this option are that additional time in between the gas collection and the analysis means a loss of vertical  resolution  in  the  sample  and  the  transportation  of  the  equipment  from  Sodankylä to Esrange  may  result  in  calibration  loss  due  to  equipment  vibrations  caused  by  transportation. The ideal scenario would be to have the FMI team being at Esrange the day before the launch despite the possibility of waiting time in case of short notice launch cancellations.  This would give margins to react to unforeseen problems such as trouble on the road due to bad weather as  well  as  missing  equipment  and/or  tools.  Furthermore,  this  would  allow  additional  time  to install and calibrate their lab equipment at Esrange prior to the launch.

The FMI team will be at Ersange the day before the launch with all the necessary equipment for analysis such as the Picarro cavity ring-down spectroscopy and calibrated dry standard gases. This is to reduce the loss in vertical resolution of the sample and the  transportation of the equipment from Sodankylä to Esrange will not be required, which saves both money and time. Furthermore, this would allow additional time to install and calibrate their lab equipment at Esrange prior to the launch.

Special consideration will be put into designing the experiment so that the recovery team can easily remove the experiment from its enclosure for possible transportation back to Esrange via helicopter. Detailed instructions will be provided on operating the detachable mechanism that will be designed. In addition, instructions will be provided to ensure that the system is completely shut down and the valves secured. Shutdown will be automated however a manual shutdown mechanism will be included should the automation fail.

The analysis results that will then be used for the post flight meeting. Further analysis will then be carried out to fully understand the data. Once a full analysis of the data has been completed there is the potential for publication of research findings.


\subsubsection{Analysis Preparation}

After completing the flushing of the Aircore, the picarro analyzer has to keep working preventing any moisture and ambient air to enter. For that reason, a calibrating gas with known concentrations will be connected to the analyzer.  By the time the AirCore is recovered, the readings of the picarro will have stabilize and the analysis will start immediately.   


