\subsection{Post Flight Activities}

To ensure minimize the length of time in which mixing of the gas occurs in the collected CAC samples, is necessary that the they  be analyzed as soon as possible after the experiment landing. Measurements will be made in partnership with the Finnish Meteorological Institute (FMI) thus it is necessary that the experiment can be transported to the measurement equipment provided by FMI. To make measurements at the landing site would be unfeasible due to the size of the gas analyzer. There are two location options for the measurements to be made:

\begin{enumerate}
    \item At the FMI in Sodankylä.
    \item At Esrange Space Centre.
\end{enumerate}

As the balloon will travel to the East, it will land closer to the FMI facility than that of Esrange. Therefore if it is possibly preferable to take the experiment to the FMI facility in order to minimize the time between the collection of the gases and the analysis of the samples. 

\textbf{Option 1:}
To facilitate this option, special consideration will be put into designing the experiment so that the recovery team can easily remove the experiment from its enclosure and pass it over to team members who will be on stand-by at the truck pick-up meeting point. Detailed instructions will be provided on operating the detachable mechanism that will be designed. 

In addition, instructions will be provided to ensure that the system is completely shut down and the valves secured. Shutdown will be automated however a manual shutdown mechanism will be included should the automation fail.

Members of the TUBULAR team associated with data analysis will need to leave for Sodankylä as soon as the balloon has been unfolded and the launch is confirmed to have occurred. Sodankylä is approximately 5 hours from Esrange which means that the team should arrive at Sodankylä approximately the same time that the balloon touches down. If updates on the flight path and predicted landing site can be given then the car will be able to redirect to the truck pick-up meeting point instead of driving all the way to Sodankylä. Once the team has met up with the recovery team the experiment can be picked up and taken to Sodankylä for analysis. This itinerary may have to be revised should location of the balloon landing site take longer than expected.

Once the samples have been analyzed the preliminary data will be sent from the team at Sodankylä back to the team at Esrange for the post-flight meeting.

\textbf{Option 2:}
If it is not possible to take the experiment to Sodankylä then the measurement equipment will be brought to Esrange by the FMI team. The FMI team would begin their travel to Esrange as soon as the ballon landing site has been located. Once the experiment has been recovered to Esrange the samples will be analyzed by the FMI equipment and the preliminary results will be found. The disadvantages of this option are that additional time in between the gas collection and the analysis means a loss of vertical resolution in the sample and the transportation of the equipment from Sodankylä to Esrange may result in calibration loss due to equipment vibrations caused by transportation. 

Either option will produce analysis results that will then be used for the post flight meeting. Further analysis will then be carried out to fully understand the data. Once a full analysis of the data has been completed there is the potential for publication of research findings.


