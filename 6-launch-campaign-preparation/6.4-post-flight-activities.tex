\subsection{Post Flight Activities}

\subsubsection{CAC Recovery}
It is important that the CAC is recovered as quickly as possible. The experiment has been designed so that the recovery team can easily remove the AirCore in the CAC box from the gondola without having to remove the entire experiment. This is to facilitate possible transportation back to Esrange via helicopter.

This quick recovery is important to minimize the length of time in which mixing of the gas occurs in the collected CAC sample. The sample should be analyzed within five to six hours after the experiment lands. At PDR it was discussed that the CAC box could be brought back to Esrange on the helicopter instead of the truck. This situation would be preferable for TUBULAR Team. 
The FMI team will arrive at Esrange one or two days before the launch with all the necessary equipment for pre-flight flushing and post-flight analysis. Having the FMI team at Esrange will give additional time for them to install and calibrate their lab equipment and also allow them to proceed faster with the analysis process as soon as the CAC is returned to Esrange. 

Detailed instructions are provided on how to remove the CAC box. In addition, instructions are provided to ensure that the system is completely shut down and the valves secured. Shutdown will be automated however, a manual shutdown mechanism will be included should the automation fail.

\subsubsubsection{Recovery Checklist}
\label{sec:recovery-checklist}

\begin{itemize}
    \item Insert the three plastic plugs to the outlet of the three tubes. The plastic plugs will be provided to the recovery team. 
    \item Unplug the gondola power cord from the AAC box. Circled with YELLOW paint.
    \item Unplug the E-Link connection from the AAC box. Circled with YELLOW paint.
    \item Unplug the D-Sub connector from the CAC Box. Circled with YELLOW paint.
    \item Unscrew 6 screws in the outside face of the experiment. Painted in YELLOW.
    \item Unscrew 6 screws in the inside face of the experiment. Painted in YELLOW.
    \item Remove the CAC Box from the gondola. Handles located at the top of the box. 
\end{itemize}

%The analysis results that will then be used for the post flight meeting. Further analysis will then be carried out to fully understand the data. Once a full analysis of the data has been completed there is the potential for publication of research findings.


\subsubsection{Analysis Preparation}

To prevent ambient air moisture entering into the analyzer, the Picarro analyzer has to keep working during the flight. After the CAC has been flushed a calibrating gas will be connected to the analyzer and keep running through it until the CAC analysis. 
The reason this is done is because it is necessary that the readings of calibrating gas stabilize before starting the analysis and the presence of moisture makes this stabilization slower. Having the analyzer running during flight saves precious time as it makes possible to start the analysis as soon as the CAC is recovered. 



% After completing the flushing of the CAC, the picarro analyzer has to keep working. For that reason a calibrating gas will be connected to the analyzer and running through it, until the analysis of the CAC, preventing ambient air moisture to enter into the analyzer.
% It is necessary that the readings of the calibating gas running through the picarro analyzer have been stabilized before analyzing the CAC sample. This will distinguish where the collected air sample starts and where it stops. 
% So, having the analyzer running during the flight will save precious time, and    Using a gas with known concentrations will make easy the distinguish between the collected samples and    By the time the CAC is recovered, the readings of the picarro will have stabilize and the analysis will start immediately.   


