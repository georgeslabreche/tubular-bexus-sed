\subsection{Preparation and Test Activities at Esrange}\label{prep_for_Esrange}
The ground station laptop PC was put in place and set up so it was operational. The communication through E-link with the experiment was tested. The air sampling schedule on the SD card was checked before flight.

In the preparation phase magnesium filters were prepared. These were short (7 cm) lengths of stainless steel tubing that were filled with 2 mg of fresh magnesium perchlorate powder \cite{Karion}. One was attached to the inlet of the CAC tubing, to ensure that no moisture entered the tubing during testing or sampling. The magnesium perchlorate powder was loosely packed to make sure that the air flow was not blocked. Stone wool was placed at both ends of the tube to prevent the powder escaping from the filter.

The same set-up was used for the AAC. As stratospheric air is dry the risk of moisture entering the system during sampling was very low, however, the team decided to use one to reduce the risk of condensation in the samples after landing. 

A few days before the flight while in Finland, the CAC was left inside an oven at 110 $\degree$C for 5 hours. At the same time, nitrogen was running through the CAC at a flow rate of 110 ml/min. This was necessary, in order to remove humidity sufficiently through evaporation. Due to the high temperature of the oven and having nitrogen running through the system, it was made sure to remove the humidity from the coil. 

Two days before the flight, the CAC  went through some preparations. At 19:44 on Sunday 14 of October, the flushing of the coiled tube with fill gas started. A fill gas is air with a spike of a known gas, for example CO. During the flushing process the coiled tube, solenoid valve, the exit tube as well as the magnesium perchlorate filter were flushed separately. In the flushing process the quick connectors at outlet and inlet were connected to the fill gas bottle and Picarro analyzer respectively. The fill gas with a flow rate of 2 L/min was then flown through the coiled tube all the way to the Picarro for 10 minutes. Then, it was left  flushing over night at a flow rate of 40.8 ml/min to ensure unknown gases inside the tube would be removed.


After approximately 11 hours, the flushing of the CAC was over at 07:03 on Monday 15 of October. When the flushing procedure was over, Picarro was disconnected while the fill gas remained connected in order to over-pressure the CAC. This ensured that when the other parts were connected ambient air would not enter the system, as the fill gas would be exiting. Leaving the CAC over-pressured also ensured that if the quick connectors were leaking it would be fill gas leaking our and not ambient air leaking in. This was important as there were two days in between flushing and flight. Meanwhile on other hand the solenoid valve and the exit tube were flushed manually for approximately 5 minutes at a flow rate of 2L/min. As a last step, the outlet and inlet were sealed while the gas was still running through the CAC and therefore the CAC was filled and over-pressured. Thereafter it was attached to the remaining components such as the magnesium perchlorate filter, solenoid valve and exit tube. At this stage the CAC was ready for the flight. 

A pre-launch checklist in Appendix \ref{sec:appL}, was made to assure that the flight preparations will be done thoroughly. This includes a step by step flushing procedure.

For the AAC system, the manifold was cleaned by flushing it with a dry gas as soon as all the pre-flight testing was done. The dry gas is extracted from the fill gas and has slightly different concentrations from the fill gas. The dry gas bottle, the vacuum pump and the AAC system, were all connected as a system to the central valve. The pump and the valves of the AAC system were cleaned by this procedure. 
The night of the flight, Tuesday 16 of October at 23:00, the flushing of the AAC started. The plugs from the inlet and outlet tubes of the AAC were unscrewed and a male thread quick connector was screwed in to the inlet tube. The dry gas, vacuum pump, and central valve system were then connected to the inlet tube. Flushing started when the central valve was open to dry gas, and dry gas started flowing into the AAC manifold while the flushing valve was open with the rest of the valves closed. The flow rate was at 2L/min and the flushing procedure was going on for approximately 15 minutes. When the central valve was closed, and dry gas stopped flowing into the AAC, the flushing valve was closed. The dry gas bottle, the vacuum pump, and the central valve system were disconnected from the inlet tube. The plug was screwed in to the inlet tube.  

In a second phase, using the AAC valves this time, the bags and consequently the tubes between the bags and the manifold, were flushed, again after the pre-flight testing was done. Only one bag was flushed at a time, using the central valve, the flushing valve, and the solenoid valve that matched the bag to control which bag was being flushed. The flushing had to be done three times for each bag to ensure the bags were properly cleaned. It was also important to flush the manifold again, between the flushing of each bag. 
The dry gas, the vacuum pump, and the central valve system were connected to the outlet tube, while the inlet tube was sealed. Next, the bags manual valves were opened. The flushing valve was kept open during the whole procedure. Only one solenoid valve that matched the bag which was being flushed was opened at a time. 

Flushing started when the central valve was open to dry gas, and dry gas started flowing through the AAC manifold, tubes, into the bag. The flow rate was set at 2 L/min. It took 1.5 minutes to fill each bag with 3L of dry gas. Then, the central valve was turned open to the vacuum allowing the bag to empty, for approximately 4.5 minutes. This procedure was repeated for all six bags. After the flushing of one bag was completed, the dry gas, vacuum pump, and central valve system was disconnected from the outlet tube and connected to the inlet tube, allowing the manifold to be flushed, before flushing the next bag, as described above. Then the dry gas, vacuum pump, and central valve system were connected to the outlet tube again, and the next bag was flushed. The whole flushing procedure took approximately 3 hours. After the end of the flushing, when the bags were empty again, the flushing valve was closed. The dry gas, vacuum pump, and central valve system was disconnected from the outlet tube and the plug was screwed in. At this point, the AAC was ready for flight.   

The pre-launch checklist in Appendix \ref{sec:appL} was again made sure that all the steps were done correctly and in the right order. 

In a laboratory phase, tests under monitored conditions were done to evaluate the overall consistency of the CAC and the AAC. In particular, the CAC and the AAC were tested for leaks at the junctions and at the valves. 

Furthermore, the team decided to clean the rest of the experiment's components, such as the Brain, as well as the structure. Doing so, any unwanted particles released during the experiment's construction, was removed avoiding these particles to enter the pneumatic system and thus contaminating the collected samples. 

The system was cleaned manually with a dust cloth, using gloves and IPA, given that this cleaning procedure is not of high need as the cleaning of the coil or the bags. Considering that the building of the experiment took place in a lab, which was a clean environment, this action was done once before the flight. This procedure was done just after EAR.


%For that reason, an appropriate device for small and sensitive components, such as a vacuum cleaner or a machine that blows air, will be used. If this is not possible, then the experiment will be cleaned manually with a dust cloth given that this cleaning procedure is not of high need as the cleaning of the coil or the bags. Considering that the building of the experiment will take place in a lab, which is a clean environment, this action will be done once before the flight, and the procedure may change if another, more effective way of cleaning is found. This procedure will be done just after EAR.
