\subsection{Preparation and Test Activities at Esrange}
The ground station laptop PC will need to be put in place and operational. The communication through E-link with the experiment shall be tested. The air sampling itinerary has to be checked before flight.

What is more, in a preparation phase, short (7 cm) lengths of stainless steel tubing, will be filled with fresh magnesium perchlorate powder \cite{Karion} and will be attached to either end of the CAC tubing, as seen in Figure \ref{fig:CAC-schematic} to ensure that no moisture will enter the tubing during any testing or sampling. The tubes will be loosely filled with magnesium perchlorate powder, making sure that the air flow is not blocked. What is more, stone wool will be placed at both ends of the tube, preventing the powder getting inside the AirCore or the bags.

Magnesium perchlorate powder will be used for the AAC as well. As seen in Figure \ref{fig:pipes_interface} the filter, in front of the valve center, will be coated with magnesium perchlorate to prevent moisture from entering while sampling. Stratospheric air is dry and it is not necessary to use a moisture filter. But the team decided to use one anyway, making sure that the samples will have no moisture, reducing the risk of condensation of the samples after landing. Again the tubes will be prepared the same way as for the AAC, with stone wool at both of its ends and loosely filled with the powder.   

Before the launch, both CAC and AAC have to go through some preparations.
In the case for CAC, the coil tube is going to be flushed and filled with a fill gas. A fill gas is basically air with a spike of a known gas, for example CO. In the Figure  \ref{fig:CAC-schematic} the quick connector at interface No 5., will be connected to the picarro analyzer and the fill gas will go through the picarro into the AirCore. The coil tube will be flushed at a flow rate of 40ml/min by leaving quick connector, No 1., at the outlet opened over night, to ensure unknown gases inside the tube will be removed. Once it is done, the outlet will be closed and the filling process will take place. Parts from No 6. to No. 8 won't be able to be flushed due to it is not necessary and might affects the amount of magnesium perchlorate powder in the tube. From part No 9. to No 13. such flushing process is not ideal for the solenoid valve, because that means it needs to be operating over night which can harm the electronic. Therefor this section will be manually flushed. When it is completed, the solenoid valve will be closed and the AirCore will be ready for the flight. Evacuation of the CAC tube will be done passively through the progressive decrease in air pressure during the balloon's ascent phase.

In the case for AAC, all the bags will be cleaned with a dry gas. The dry gas is extracted from the fill gas and has slightly different concentrations. The dry gas bottle, the vacuum pump and one bag at a time, will all be connected as a system to a central valve. Its position will determine whether the dry gas or the vacuum is open, depending on whether the bag is being filled or emptied. The bag's valve will remain open during the whole flushing procedure. The filling of the bag will be done with the central valve in the position where the vacuum pump is closed and the dry gas can flow into the bag. The next stage is to place the central valve where the vacuum pump is open and the dry gas is blocked, allowing the bag to empty. The flushing has to be done three times for each bag. A flow sensor will be placed close to the central valve, making sure that all bags will be filled with the same amount of dry gas, which is the same as the bag's capacity, 3L. It is preferred to start the flushing process for each bag with a lower flow rate and increasing it for the second and third time. After the end of this procedure, each bag will be sealed, by closing its valve, while the vacuum pump is open.

In addition, for the AAC case, the tubes between the bags and the manifold have to be flushed with the dry gas as well. The bags will be already flushed and sealed and the flushing procedure of the tubes will not affect the bags. Again, only one tube will be flushed at a time, with the inlet and outlet valves, as well as the T-union valve and the corresponding to the tube solenoid valve in the manifold, open. The tubes have to be flushed for as much time as it takes until the dry gas amount that goes through them is ten times theirs volume. This is going to be controlled with an air flow sensor. After the flushing is complete the T-union will be closed, as well as the corresponding solenoid valve in the manifold. The dry gas now will be connected to the next tube and the same procedure will follow. After the flushing of all the tubes is complete, and all the T-union valves and the solenoids valves in the manifold are closed, the pump will be started with the inlet and outlet valves open, to flush the system. At the end, the pump will be shut off, the outlet valve will be closed, and the AAC will be ready for flight.   

In a laboratory phase, tests under monitored conditions will be done to evaluate the overall consistency of the CAC and the AAC. In particular, the CAC and the AAC shall be tested for leaks at the junctions and at the valves. 

Should a gas chromatograph be made available on site, pre-flight testing will be made to ensure sample concentration preservation. To do so, calibrated dry standard gases of two different values for both CO$_2$ and CH$_4$ will be used. One with high CO$_2$ and CH$_4$ concentrations, i.e high-concentration calibration standard, and one with lower concentrations, i.e low-concentration standard. Should a gas chromatograph not be available, this activity will be incorporated in the latter part of the Experiment Building and Testing stage.