\subsection{Preparation and Test Activities at Esrange}\label{prep_for_Esrange}
The ground station laptop PC will need to be put in place and set up so it is operational. The communication through E-link with the experiment shall be tested. The air sampling schedule on the SD card has to be checked before flight.

In the preparation phase magnesium filters will be prepared. These are short (7 cm) lengths of stainless steel tubing that will be filled with fresh magnesium perchlorate powder \cite{Karion}. One will be attached to the inlet of the CAC tubing, to ensure that no moisture enters the tubing during testing or sampling. The magnesium perchlorate powder will be loosely packed to make sure that the air flow is not blocked. Stone wool will be placed at both ends of the tube to prevent the powder escaping from the filter.

The same set-up will be used for the AAC. As stratospheric air is dry the risk of moisture entering the system during sampling is very low however, the team decided to use one to reduce the risk of condensation in the samples after landing. 

The day before the flight, both CAC and AAC have to go through some preparations. For the CAC, the coiled tube will be flushed and filled with a fill gas. A fill gas is air with a spike of a known gas, for example CO. During the flushing process the coiled tube and solenoid valve and the exit tube will be flushed separately. The magnesium perchlorate filter will not be flushed because it is not necessary. In the flushing process the quick connectors at outlet and inlet will be connected to the fill gas bottle and Picarro analyzer respectively. A fill gas will then flow through the coiled tube all the way to the Picarro. It will be flushed over night at a flow rate of 40 ml/min to ensure unknown gases inside the tube will be removed. Meanwhile on other hand the solenoid valve and the exit tube will be flushed manually. The outlet and inlet will be sealed while the gas is still running through the CAC and therefore the CAC will be filled. Thereafter it will be attached to remaining components such as the magnesium perchlorate filter, solenoid valve and exit tube. At this stage the CAC is ready for the flight. 

A pre-launch checklist in Appendix \ref{sec:appL}, was made to assure that the flight preparations will be done thoroughly. 

For the AAC all the bags will be cleaned by flushing them with a dry gas the night before the flight. The dry gas is extracted from the fill gas and has slightly different concentrations from the fill gas. The dry gas bottle, the vacuum pump and the AAC system, will all be connected as a system to the central valve. Using the AAC valves and the bags manual valve to control which bags are sealed and which are opened the process will go through each bag individually. The central valve's position will determine whether the dry gas or the vacuum is open. When the valve is open to dry gas the sampling bags are being filled and when it is open to vacuum they are being emptied. The flushing has to be done three times for each bag to ensure the bags are properly cleaned. A flow sensor will be placed close to the central valve, making sure that all bags will be filled with 3 L of dry gas. After the end of this procedure, when the bags are empty again, each bag will be sealed, by closing its valve, while the vacuum pump is open. The pump and the valves of the AAC system will also be cleaned by this procedure. It is anticipated this process will take around 4 hours to complete.

Once the sampling bags have been cleaned and sealed, the system of tubes between the bags and the manifold have to be flushed with the dry gas as well. Again, only one tube will be flushed at a time, using the central valve, the T-union and the solenoid valve that matches the tube to control which tube is being flushed. The tubes have to be flushed until ten times their volume has passed through, this will be monitored with the air flow sensor. When flushing is complete the dry gas connection will be removed from the T-union and the T-union will be automatically closed (quick connector interface). The corresponding solenoid valve shall be closed at the same time the fill gas is disconnected. The dry gas is then connected to the next tube and the same procedure follows. After the flushing of all the tubes is complete, and the system is sealed, the pump will be started with the flushing valve open, to flush the AAC system. At the end, the pump will be shut off, the flushing valve will be closed, and the AAC will be ready for flight. Note that the manual valves of the sampling bags have to be opened before flight.    

The pre-launch checklist in Appendix \ref{sec:appL} will again make sure that all the steps will be done correctly and in the right order. 

In a laboratory phase, tests under monitored conditions will be done to evaluate the overall consistency of the CAC and the AAC. In particular, the CAC and the AAC shall be tested for leaks at the junctions and at the valves. 

What is more, the team has decided to clean the rest of the experiment's components, such as the Brain, as well as the structure. Doing so, any unwanted particles released during the experiment's construction, will be removed avoiding these particles to enter the pneumatic system and thus contaminating the collected samples. 
For that reason, an appropriate device for small and sensitive components, such as a vacuum cleaner or a machine that blows air, will be used. If this is not possible, then the experiment will be cleaned manually with a dust cloth, considering that this cleaning procedure is not of high need as the cleaning of the coil or the bags. What is more, the building of the experiment will take place in a lab, which is a clean environment. Therefore this action will be done once before the flight, and the procedure may change if another, more effective way of cleaning is found.
It is yet to be decided whether this procedure will be done the day before the flight or some day earlier.
