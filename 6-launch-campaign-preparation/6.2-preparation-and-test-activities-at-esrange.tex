\subsection{Preparation and Test Activities at Esrange}
The ground station laptop PC will need to be put in place and operational. The communication through E-link with the experiment shall be tested. The air sampling itinerary has to be checked before flight.

What is more, in a preparation phase, short (15 cm) lengths of stainless steel tubing, will be filled with fresh magnesium perchlorate powder \cite{Karion} and will be attached to either end of the CAC tubing, to ensure that no moisture will enter the tubing during any testing or sampling.

Magnesium perchlorate powder will be used for the AAC. As seen in Figure \ref{fig:pipes_interface} the filter, in front of the valve center, will be coated with magnesium perchlorate to prevent moisture from entering while sampling. 

Before the launch, both CAC and AAC will be filled with  an inert gas. All the bags will be manually emptied before flight, while the CAC will empty itself  during ascent phase.  
For the experiment will be used nitrogen, which is referred to as and used as an inert gas. Nitrogen's triple bond is very strong and requires a lot of energy to break those bonds and participate in a reaction. That is why nitrogen is used, despite the fact that is not truly inert like most noble gases.

In a laboratory phase, tests under monitored conditions will be done to evaluate the overall consistency of the CAC and the AAC. In particular, the CAC and the AAC shall be tested for leaks at the junctions and at the valves. 

Should a gas chromatograph be made available on site, pre-flight testing will be made to ensure sample concentration preservation. To do so, calibrated dry standard gases of two different values for both CO$_2$ and CH$_4$ will be used. One with high CO$_2$ and CH$_4$ concentrations, i.e high-concentration calibration standard, and one with lower concentrations, i.e low-concentration standard. Should a gas chromatograph not be available, this activity will be incorporated in the latter part of the Experiment Building and Testing stage.