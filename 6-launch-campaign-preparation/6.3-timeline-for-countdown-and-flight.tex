\subsection{Timeline for Countdown and Flight}
Table \ref{tab:countflight} was the estimated timeline during countdown and flight. 

The desired altitudes in which air samples were to be collected with the sampling bags was associated with specific air pressure values. Thus, the valve operations to sample air during the balloon Ascent and Descent Phases were to be triggered by readings from the ambient pressure sensor. The time values presented in Table \ref{tab:countflight} merely served as an indicative estimate of when the sampling will take place as sampling was not programmed based on flight time.

\begin{table}[H]
\centering


\begin{tabular}{|l|l|m{0.6\textwidth}|}
\hline
\multicolumn{1}{|c|}{\textbf{Time}}       & \multicolumn{1}{c|}{\textbf{Altitude}}      & \multicolumn{1}{c|}{\textbf{Events}}                              \\ \hline
\multicolumn{1}{|c|}{T-1/2DAYS}    & \multicolumn{1}{c|}{0}             & Start flushing the CAC system overnight for 8H                           \\ \hline
\multicolumn{1}{|c|}{T-7H}    & \multicolumn{1}{c|}{0}             & Start flushing the AAC system for 3H                                \\ \hline
\multicolumn{1}{|c|}{T-3H}    & \multicolumn{1}{c|}{0}             & Experiment is switched on external power                                \\ \hline
\multicolumn{1}{|c|}{T-3H}    & \multicolumn{1}{c|}{0}             & Experiment goes to Standby mode                          \\ \hline
\multicolumn{1}{|c|}{T-1H}    & \multicolumn{1}{c|}{0}             & Experiment switches to internal power                                \\ \hline
\multicolumn{1}{|c|}{T=0}        & \multicolumn{1}{c|}{0}             & Lift-off                                                 \\ \hline
\multicolumn{1}{|c|}{T+1s}       & \multicolumn{1}{c|}{$\sim$5 meter} & Experiment goes to Normal - Ascent mode                  \\ \hline
\multicolumn{1}{|c|}{T+15 min}   & \multicolumn{1}{c|}{1 km}          & Experiment starts to empty the CAC's tube\\ \hline
%T+45 min                         & 15 km                              & Experiment stops emptying the tubes              \\ \hline
\multicolumn{1}{|c|}{T+$\sim$1H} & \multicolumn{1}{c|}{$\sim$18 km}   & Take air samples with AAC until $\sim$24 km                       \\ \hline
T+$\sim$1.5H                     & $\sim$25 km                        & Float Phase                                           \\ \hline
T+$\sim$2.5H                     & $\sim$25 km                        & Cut-off                                                  \\ \hline
T+$\sim$2.6H                     & $\sim$25 km                        & Experiment goes to Normal - Descent mode                 \\ \hline
T+$\sim$2.75H                    & $\sim$20 km                        & Parachute is deployed                                    \\ \hline
T+$\sim$2.8H                     & $\sim$19 km                        & Take air samples with AAC and CAC until 10 km above ground                 \\ \hline
T+3.5H                           & $\sim$10 km                         & Experiment goes to SAFE mode (all valves are closed)                            \\ \hline
\end{tabular}
\caption{Countdown and Flight Estimated Timeline.}
\label{tab:countflight}
\end{table}
\raggedbottom

Table \ref{tab:LaunchTimelineActuall} shows the actual timeline which occurred during flight. The in-flight pump startup failure that occurred is shown together with the relevant actions taken during the in flight analysis of what the problem might have been. The procedure of differential pressure difference is also shown. After attempting to start the pump several times the team recognised already that a likely cause of failure was related to the pump getting enough current, therefore several different procedures were attempted to start the pump. The first was attempting to turn the pump on when everything other than the Arduino was switched off. The second was heating the pump up until the temperature readings showed that the pump was near the top of its operating range and then attempting to turn it on, still with all other components except the Arduino turned off. Neither of these attempts worked. A third idea was to try and start it by creating a pressure difference during the descent, however this was not attempted as it risked the CAC sample. Instead during descent the valves were opened to attempt passive sampling of the bags. However due to the lack of pressure difference between the bag and the ambient pressure this had a low probability of success. 



\begin{longtable}{|m{0.1\textwidth}|m{0.15\textwidth}|m{0.7\textwidth}|} 
    \hline
        \textbf{Time} & \textbf{Altitude} & \textbf{Event}\\ \hline
        T-03:54 & 0 & Go to manual mode for 6 seconds then back to standby\\ \hline
        T-03:07 & 0 & Restart Ground Station\\ \hline
        T-03:02 & 0 & Groundstation Back Online and reciving data\\ \hline
        T-00:00 & 0 & Liftoff and automatic Accent mode \\ \hline
        T+00:31 & 10.1-10.3 km & Pump heater on 12 minutes\\ \hline
        T+00:58 & 18.8-19.1 km & First flushing was suposed to start, instead the Arduino resets and resulting in CAC valve closing \\ \hline
        T+00:58 & 18.8-19.1 km & Enter Manual Mode\\ \hline
        T+01:01 & 19.7-20 km & Flushing Valve opens for 20 seconds\\ \hline
        T+01:05 & 20.9-21.2 km & CAC Valve Reopened\\ \hline
        T+01:06 & 21.2-21.6 km & Pump Heater is on for 6 minutes \\ \hline
        T+01:11 & 22.8-23.1 km & Board Resets Due to attempting to start pump\\ \hline
        T+01:12 & 23.1-23.5 km & Pump Heater is on for 4 minutes \\ \hline
        T+01:17 & 24.4-24.5 km & Board resets due to attempt to start pump and open flushing Valve\\ \hline
        T+01:20 & 24.4-24.4 km & Flush Valve and Valve 1 is turned on to check if it would induce an error \\ \hline
        T+01:21 & 24.4-24.5 km & Float Phase Entered\\ \hline
        T+01:32 & 24.5-24.5 km & Change scheduler to 1 - 2 mbar for all bags as an atempt to make sure the system would not attempt to automaticaly sample \\ \hline
        T+01:36 & 24.4-24.4 km & Pump and Valve Heater on \\ \hline
        T+02:10 & 24.3-24.4 km & Pump and Valve Heater off\\ \hline
        T+02:10 & 24.3-24.4 km & Attempt to start Pump, fails and resets board\\ \hline
        T+02:12 & 24.3-24.3 km & CAC Valve opens and prepare for decent\\ \hline
        T+02:55 & 24.1-24.1 km& Decent Phase Entered\\ \hline
        T+03:04 & 14.9-14.1 km & Valve 2 is opened in an attempt to fill bags with ambient pressure difference \\ \hline
        T+03:09 & 11.0-10.4 km & Valve 2 is closed \\ \hline
        T+03:10 & 10.4-9.8 km & Valve 3 is opened in an attempt to fill bags with ambient pressure difference \\ \hline
        T+03:14 & 7.9-7.3 km & Valve 3 i closed \\ \hline
        T+03:15 & 7.3-6.7 km & Valve 4 i opened in an attempt to fill bags with ambient pressure difference \\ \hline
        T+03:16 & 6.7-6.2 km & Valve 4 is closed \\ \hline
        T+03:16 & 6.7-6.2 km & Accidental closing of CAC valve to early \\ \hline
        T+03:17 & 6.2-5.7 km & CAC Valve accidentally reopens for 1 second. \\ \hline
    \caption{TUBULAR BEXUS 26 Launch Timeline}
    \label{tab:LaunchTimelineActuall}
\end{longtable}
    