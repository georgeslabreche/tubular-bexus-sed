\begin{table}[H]
\centering

\begin{tabular}{|m{0.3\textwidth}| m{0.7\textwidth} |}
\hline
\textbf{Test Number} & 30 \\ \hline
\textbf{Test Type} & Verification \\ \hline
\textbf{Test Facility} & IRF / Kiruna Space Campus \\ \hline
\textbf{Tested Item} & Sampling Bags \\ \hline
\multirow{2}{*}{\textbf{\begin{tabular}[c]{@{}l@{}}Test Level/ Procedure \\ and Duration\textsuperscript{\ref{fn:testing}}\end{tabular}}} & Continuously pump air into the sampling bags until the sampling bags burst. If the tested sampling bag does not burst after 3 minutes of continuous pumping, remove the sampling bag from the pressure chamber and leave at rest to check if it will burst within 48 hours. If bursting occurs in the chamber while the sampling bag is being pump then observe and characterize its impact to assess whether a similar bursting risks damaging the sampling bag's surrounding in the experimental setup. If the bursting occurs during the 48 hours rest period then observe and characterize the damage/rupture on the sampling bag to assess whether a similar bursting risks damaging the sampling bag's surrounding in the experimental setup. \\ & Test duration: 3 minutes to 48 hours. \\ \hline
\textbf{Test Campaign Duration} & 3 days \\ \hline
\textbf{Test Campaign Date} & 1st, 2nd and 4th May \\ \hline
\textbf{Test Completed} & YES \\ \hline
\end{tabular}
\caption{Test 30: Bag bursting test description}
\label{tab:bag-burst}
\end{table}


\raggedbottom

% \begin{table}[H]
% \centering

% \begin{tabular}{|m{0.3\textwidth}| m{0.7\textwidth} |}
% \hline
% \textbf{Test Number} & 31 \\ \hline
% \textbf{Test Type} & Verification \\ \hline
% \textbf{Test Facility} & IRF / Kiruna Space Campus \\ \hline
% \textbf{Tested Item} & Sampling Bags \\ \hline
% \multirow{2}{*}{\textbf{\begin{tabular}[c]{@{}l@{}}Test Level/ Procedure \\ and Duration\textsuperscript{\ref{fn:testing}}\end{tabular}}} & Continuously pump air into the sampling bags at lowest and highest predicted air pressures until the sampling bags burst. If the tested sampling bag does not burst after 3 minutes of continuous pumping, remove the sampling bag from the pressure chamber and leave at rest to check if it will burst within 48 hours. If bursting occurs in the chamber while the sampling bag is being pump then observe and characterize its impact to assess whether a similar bursting risks damaging the sampling bag's surrounding in the experimental setup. If the bursting occurs during the 48 hours rest period then observe and characterize the damage/rupture on the sampling bag to assess whether a similar bursting risks damaging the sampling bag's surrounding in the experimental setup. \\ & Test duration: 3 minutes to 48 hours. \\ \hline
% \textbf{Test Campaign Duration} & 3 days \\ \hline
% \textbf{Test Campaign Date} & May \\ \hline
% \textbf{Test Completed} & YES \\ \hline
% \end{tabular}
% \caption{Test 31: Bag bursting test description}
% \label{tab:vacuum-test}
% \end{table}


% \raggedbottom