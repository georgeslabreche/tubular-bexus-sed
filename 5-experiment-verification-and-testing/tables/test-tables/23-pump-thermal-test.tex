\begin{table}[H]
\centering

\begin{tabular}{|m{0.3\textwidth}| m{0.7\textwidth} |}
\hline
\textbf{Test Number} & 23 \\ \hline
\textbf{Test Type} &- \\ \hline
\textbf{Test Facility} &- \\ \hline
\textbf{Tested Item} & - \\ \hline
\multirow{2}{*}{\textbf{\begin{tabular}[c]{@{}l@{}}Test Level/ Procedure \\ and Duration\end{tabular}}} & Test procedure: - \\ & Test duration:- \\ \hline
\textbf{Test Campaign Duration} & - \\ \hline
\textbf{Test Campaign Date} & - \\ \hline
\textbf{Test Completed} & -\\ \hline
\end{tabular}
\caption{Test 23: REMOVED - COMBINED WITH 5.}
\label{tab:pump-thermal-test}
\end{table}


\raggedbottom

% \begin{table}[H]
% \centering

% \begin{tabular}{|m{0.3\textwidth}| m{0.7\textwidth} |}
% \hline
% \textbf{Test Number} & 23 \\ \hline
% \textbf{Test Type} & Thermal \\ \hline
% \textbf{Test Facility} & Esrange Space Centre TBC \\ \hline
% \textbf{Tested Item} & Pump \\ \hline
% \multirow{2}{*}{\textbf{\begin{tabular}[c]{@{}l@{}}Test Level/ Procedure \\ and Duration\textsuperscript{\ref{fn:testing}}\end{tabular}}} & Test procedure: The pump will be placed in a temperature chamber and tested from +40 to at least -40 and its operation will be recorded. This test will happen once to simulate the operation of the pump during testing where it will cycle between on and off. The test will be repeated but this time the pump will be off for long enough that the internal temperature of the pump is the same as the ambient temperature of the chamber. Then it will be attempted to turn the pump on. This is to simulate the reaction of the pump to being off during the Float phase. \\ & Test duration: 2 days \\ \hline
% \textbf{Test Campaign Duration} & 1 weeks \\ \hline
% \textbf{Test Campaign Date} & August \\ \hline
% \textbf{Test Completed} & NO \\ \hline
% \end{tabular}
% \caption{Test 23: REMOVED - COMBINED WITH 5}
% \label{tab:pump-thermal-test}
% \end{table}


% \raggedbottom