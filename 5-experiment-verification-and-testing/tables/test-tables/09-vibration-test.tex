
\begin{table}[H]
\centering

\begin{tabular}{|m{0.3\textwidth}| m{0.7\textwidth} |}
\hline
\textbf{Test Number} & 9 \\ \hline
\textbf{Test Type} & Vibration \\ \hline
\textbf{Test Facility} & IRF/LTU, Kiruna \\ \hline
\textbf{Tested Item} & Entire experiment \\ \hline
\multirow{2}{*}{\textbf{\begin{tabular}[c]{@{}l@{}}Test Level/ Procedure \\ and Duration\end{tabular}}} & Test procedure: Mount the experiment on the back of a car/trailer in the same way it will be mounted on the gondola and drive over a bumpy or rough terrain. Afterwards, check the experiment for functionality and structural integrity.\\ & Test duration: 2 hours \\ \hline
\textbf{Test Campaign Duration} & 1 week \\ \hline
\textbf{Test Campaign Date} & 3rd - 7th September \\ \hline
\textbf{Test Completed} & YES \\ \hline
\end{tabular}
\caption{Test 9: Vibration Test Description.}
\label{tab:vibration-test}
\end{table}

\raggedbottom

%Use a shake table in the university facilities to test both random and sinusoidal vibrations. The boxes will be tested individually and attached together. In order to inspect the response of the inside elements, the test will also be done without the walls.