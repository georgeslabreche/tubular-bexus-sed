\begin{table}[H]
\centering

\begin{tabular}{|m{0.3\textwidth}| m{0.63\textwidth} |}
\hline
\textbf{Test Number} & 20 \\ \hline
\textbf{Test Type} & Electronics \\ \hline
\textbf{Test Facility} & LTU, Kiruna \\ \hline
\textbf{Tested Item} & Valves, Arduino, Switching Circuit \\ \hline
\multirow{2}{*}{\textbf{\begin{tabular}[c]{@{}l@{}}Test Level/ Procedure \\ and Duration\end{tabular}}} & Test procedure: Beginning on a bread board the switching circuit will be set up connecting one end to a 3.3 V supply and another to a 24 V supply. It will be checked that turning the 3.3 V supply on and off also turns the valve/heater/pump on and off. The current draws during switching will also be monitored to check that they are in line with what the DC-DC/gondola power that can be provided. Once the circuit is working in this configuration the 3.3V supply will be switched for the Arduino and the 24 V supply to the DC-DC and the test repeated. When the circuit is working on bread board it can then be soldered onto the PCB. As it is soldered onto the PCB each switch should be checked. Finally once all switches are soldered onto the PCB a check should be made on the whole switching system that it turns on and off all components on command. \\ & Test duration: Recurrent \\ \hline
\textbf{Test Campaign Duration} & 2 months \\ \hline
\textbf{Test Campaign Date} & July and August \\ \hline
\textbf{Test Completed} & YES \\ \hline
\end{tabular}
\caption{Test 20: Switching Circuit Testing and Verification.}
\label{tab:switching-test}
\end{table}


\raggedbottom