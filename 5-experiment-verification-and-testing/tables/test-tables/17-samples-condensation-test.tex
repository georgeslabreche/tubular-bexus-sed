%\renewcommand\thempfootnote{\arabic{mpfootnote}}

\begin{table}[H]
\centering
\begin{minipage}{\textwidth}
\begin{tabular}{|m{0.3\textwidth}| m{0.63\textwidth} |}
\hline
\textbf{Test Number} & 17 \\ \hline
\textbf{Test Type} & Verification \\ \hline
\textbf{Test Facility} & FMI  \\ \hline
\textbf{Tested Item} & Sampling bags \\ \hline
\multirow{2}{*}{\textbf{\begin{tabular}[c]{@{}l@{}}Test Level/ Procedure \\ and Duration\end{tabular}}} & Test procedure: All valves, bags and tubes had to be connected. Then the entire system was flushed the same way it will be for the flight. After flushing, the sampling bags were filled with a gas of known concentration. The bags were then left outside for 6, 14, 24 and 48 hours. In total 8 sampling bags were used with two bags for each time duration. After each time duration two bags were removed and analyzed using the Picarro analyzer. The second time the test was repeated 6 sampling bags were tested and left outside for 15, 24, and 48 hours.  The concentration of gases found inside the bags were compared to the initial concentration of the air placed in the bags. If the concentration changes then the sampling bags must be retrieved and analyzed before that amount of time has elapsed for the samples to be preserved. \\ & Test duration: 3 days. \\ \hline
\textbf{Test Campaign Duration} & 5 days \\ \hline
\textbf{Test Campaign Date} & 7th-9th May AND 3rd-7th September \\ \hline
\textbf{Test Completed} & YES\\ \hline
\end{tabular}
\caption{Test 17: Sampling Bags' Holding Times.}
\label{tab:samples-condensation-test}
\end{minipage}
\end{table}
\raggedbottom