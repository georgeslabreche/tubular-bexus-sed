%\renewcommand\thempfootnote{\arabic{mpfootnote}}

\begin{table}[H]
\centering
\begin{minipage}{\textwidth}
\begin{tabular}{|m{0.3\textwidth}| m{0.7\textwidth} |}
\hline
\textbf{Test Number} & 17 \\ \hline
\textbf{Test Type} & Verification \\ \hline
\textbf{Test Facility} & FMI  \\ \hline
\textbf{Tested Item} & Sampling bags \\ \hline
\multirow{2}{*}{\textbf{\begin{tabular}[c]{@{}l@{}}Test Level/ Procedure \\ and Duration\end{tabular}}} & Test procedure: All valves, bags and tubes must be connected. Then the entire system needs to be flushed the same way it will be for the flight. After flushing, the bags will then be filled with a gas of known concentration. The bags will then be left outside for 6, 14, 24 and 48 hours. Using 8 bags in total with two bags for each time duration. After each time duration two bags will be removed and analyzed using the Picarro analyzer. The concentration of gases found inside the bags will be compared to the initial concentration of the air placed in the bags. If the concentration changes then the sampling bags must be retrieved and analyzed before that amount of time has elapsed for the samples to be preserved. \\ & Test duration: 3 days. \\ \hline
\textbf{Test Campaign Duration} & 5 days \\ \hline
\textbf{Test Campaign Date} & 7th-9th May AND 3rd-7th September \\ \hline
\textbf{Test Completed} & TO BE REPEATED\\ \hline
\end{tabular}
\caption{Test 17: Sampling Bags' Holding Times and Samples' Condensation Verification.}
\label{tab:samples-condensation-test}
\end{minipage}
\end{table}
\raggedbottom